\section{Exercício 3} 

O número máximo X de arestas de um grafo de $n$ vretices pode ser calculado da seguinte forma:

Cada vérttice pode se ligar a $n - 1$ vértices. Logo:

\begin{equation}
	X = n \cdot (n -1)
\end{equation}

Porém, precisamos apenas da metade desse número, uma vez que só deve haver uma aresta ligando dois vértices.
Logo, buscamos o seguinte valor de $X$:

\begin{equation}
	X = \frac{n \cdot (n - 1)}{2}
\end{equation}

