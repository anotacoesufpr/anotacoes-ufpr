\section{Exercício 10} 

Para responder a esta questão, utilizaremos a seguinte regra: \\

$\sum_{u \in V(G)} d(u) = 2 \cdot |E(G)|$ \\

\noindent
Ou seja, a soma dos graus de todos os vértices de G é igual ao dobro do seu número de arestas. \\


\subsection{Grafo com vértices de graus 2, 3, 3, 4, 4, 5}

Soma de graus: $2 + 3 + 3 + 4 + 4 + 5 = 21$

\noindent
Como 21 é ímpar, esses vértices não foram um garfo, já que a soma deve reslar em número par ($2 \cdot |E(G)|$)



\subsection{Grafo com vértices de graus 2, 3, 4, 4, 5}


Soma dos graus: $2 + 3 + 4 + 4 + 5 = 18$

\noindent
Como 18 é par, concluímos que os vérices podem formar um grafo de $\frac{18}{2} = 9$ arestas.
