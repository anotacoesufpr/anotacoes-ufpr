\documentclass{abnt}

\usepackage[utf8]{inputenc}
\usepackage[brazil]{babel}
\usepackage{hyperref}
\usepackage[alf]{abntcite}
\usepackage{amsmath}

\setcounter{chapter}{1}

\titulo{Segurança em Redes de Computadores}
\comentario{Segunda avaliação da disciplina de Orientação Bibliográfica}
\autor{Roberto Beraldo Chaiben}
\local{Curitiba}
\instituicao{Universidade Federal do Paraná}
\data{24/06/2010}


\begin{document}


\folhaderosto

\sumario

\chapter{Revisão de Literatura}

\section{Introdução}

As redes de computadores estão sendo utilizadas cada vez mais intensamente em todo o planeta. Além das redes locais, há a Internet,
a rede mundial de computadores. Logo, segurança é essencial, e é objeto de estudos de diversos profissionais da Tecnologia da Informação.

Com o progresso rápido e vertiginoso da computação em rede, muitas brechas de segurança foram geradas. Isso exige estudos exaustivos a fim
de detectá-las e corrigi-las, enquanto outros indivíduos exploram essas falhas com o intuito de obter informações privilegiadas
e sigilosas, para, de alguma forma, beneficiarem-se, muitas vezes ilegalmente, com elas.


\section{\textit{Backup}}

Segundo \citeonline{b20}, há duas regras iniciais para a segurança de uma rede: possuir \textit{backup} e, mais que isso, possuir mais de
um \textit{backup}, ou seja, possuir redundância de dados. Isso garante que, se informações importantes forem perdidas, será possível
recuperá-las com facilidade. Essas cópias devem ser atualizadas frequentemente, a fim de evitar o armazenamento de informações obsoletas.


\section{Segurança Física}

Segundo \citeonline{b2}, \citeonline{b15}, \citeonline{b7}, \citeonline{b12}, \citeonline{b17} e \citeonline{b20}, é imprescindível que as
máquinas essenciais para o funcionamento de uma rede estejam em um local adequado, seguro, longe de condições climáticas e ambientais que
possam prejudicar o funcionamento do sistema. Esse local também deve possuir acesso restrito a pessoas de confiança, responsáveis pela
administração da rede, impedindo que pessoas maliciosas e mal intencionadas obtenham acesso físico aos componentes vitais de uma rede.

É importante possuir geradores de energia elétrica e \textit{no-breaks}, que possam garantir o funcionamento dos principais equipamentos
da rede durante uma falta de energia.


\section{Criptografia}

Segundo \citeonline{b20}, \citeonline{b15}, \citeonline{b19}, \citeonline{b10}, \citeonline{b8} e \citeonline{b11}, criptografia consiste
na codificação de uma mensagem por meio de uma chave secreta de criptografia, ou seja, transformar uma mensagem em algo que somente quem
conheça essa chave possa compreendê-la.


\section{Algoritmos de Chave Pública}

\begin{quotation}

Em 1976, dois pesquisadores da \textit{University of Stanford}, Diffie e Hellman (1976), propuseram um
sistema de criptografia radicalmente novo, no qual as chaves de criptografia e de descriptografia
eram diferentes, e a chave de descriptografia não podia ser derivada da chave de criptografia. Em
sua proposta, o algoritmo de criptografia (chaveado) E e o algoritmo de descriptografia (chaveado)
D tinham de atender a três requisitos, que podem ser declarados da seguinte forma:

\begin{itemize}

	\item D(E(P)) = P.
    \item É extremamente difícil deduzir D a partir de E.
    \item E não pode ser decifrado por um ataque de texto simples escolhido.

\end{itemize}

(TANENBAUM, A. S.; Redes de Computadores, 2003, p. 565)

\end{quotation}

O primeiro requisito diz que, se aplicarmos D a uma mensagem criptografada, E(P), iremos obter outra vez a mensagem de texto simples original P.
Sem essa propriedade, o destinatário legítimo não poderia decodificar o texto cifrado. O segundo é autoexplicativo. O terceiro é necessário
porque os intrusos podem experimentar o algoritmo até se cansarem (\textit{brute force}). Sob essas condições, não há razão para a chave
criptográfica não se tornar pública.

O método funciona da seguinte forma: uma pessoa, digamos Maria, desejando receber mensagens secretas, primeiro cria dois algoritmos que atendem
aos requisitos anteriores. O algoritmo de criptografia e a chave de Maria se tornam públicos. Por exemplo, Maria poderia colocar sua chave pública
na \textit{home page} que ela tem na \textit{Web}. Usaremos a notação $E_M$ para indicar o algoritmo de criptografia parametrizado pela chave pública
de Maria. De modo análogo, o algoritmo de descriptografia (secreto) parametrizado pela chave privada de Maria é $D_M$. João faz o mesmo, publicando
$E_J$, mas mantendo secreta a chave $D_J$.


\subsection{RSA}

\begin{quotation}

Devido às vantagens potenciais da criptografia de chave pública, muitos pesquisadores estão se dedicando integralmente a seu estudo, e alguns algoritmos
já foram publicados. Um método muito interessante foi descoberto por um grupo de pesquisadores do \textit{MIT}\footnote{http://web.mit.edu} (\textit{Massachusetts Institute of Technology})
e é conhecido pelas iniciais dos três estudiosos que o criaram (\textit{Rivest, Shamir, Adleman}): \textit{RSA}. Ele sobreviveu a todas as tentativas de rompimento por mais de um quarto de século e é reconhecido como um
algoritmo muito forte. Grande parte da segurança prática se baseia nele. Sua principal desvantagem é exigir chaves de pelo menos 1024 \textit{bits} para manter um
bom nível de segurança (em comparação com 128 \textit{bits} para os algoritmos de chave simétrica), e isso o torna bastante lento. (TANENBAUM, A. S.; Redes de Computadores, 2003, p. 566)

\end{quotation}


O algoritmo para criptografia RSA é:

\begin{enumerate}

	\item Escolha dois números primos extensos, $p$ e $q$ (geralmente, de 1024 \textit{bits}).
	\item Calcule $n = p$ $q$ e $z = (p - 1) (q - 1)$.
	\item Escolha um número $d$ tal que $z$ e $d$ sejam primos entre si.
	\item Encontre e de forma que e $d = 1$ mod $z$.

\end{enumerate}


A segurança do método se baseia na dificuldade de fatorar números extensos. Se pudesse fatorar o valor n (publicamente conhecido), o criptoanalista poderia
então encontrar $p$ e $q$ e, a partir desses, encontrar $z$. Com o conhecimento de $z$ e $e$, é possível encontrar $d$ utilizando-se o algoritmo de
Euclides\footnote{Em Matemática, o algoritmo de Euclides é um método simples e eficiente de encontrar o máximo divisor comum entre dois números inteiros diferentes de zero}.
Felizmente, os matemáticos têm tentado fatorar números extensos por pelo menos 300 anos, e o conhecimento acumulado sugere que o problema é extremamente difícil.


\section{\textit{Hashes}}

Um \textit{hash} é uma sequência de \textit{bits} geradas por um algoritmo de dispersão, em geral representada em base hexadecimal, que permite a visualização em letras e números
(0 a 9 e A a F), representando meio \textit{byte} cada.

Essa sequência busca identificar um arquivo ou informação unicamente. Por exemplo, uma mensagem de correio eletrônico, uma senha, uma chave criptográfica ou mesmo um arquivo.
É um método para transformar dados de tal forma que o resultado seja quase exclusivo. Além disso, funções usadas em criptografia garantem que não é possível a partir de um
valor de \textit{hash} retornar à informação original, exceto utilizando-se de métodos de \textit{brute force}.


\subsection{MD5}

O MD5 é o quinto de uma série de sumários de mensagens criadas por Ronald Rivest. Ele opera desfigurando os \textit{bits} de uma forma tão complicada que todos os \textit{bits} de saída são
afetados por todos os \textit{bits} de entrada. Resumindo, a função começa aumentando o tamanho da mensagem até chegar a 448 \textit{bits} (módulo 512). Em seguida, o tamanho original é
anexado como um inteiro de 64 \textit{bits}, a fim de gerar uma entrada total cujo tamanho seja um múltiplo de 512 \textit{bits}. A última etapa antes dos cálculos serem efetuados é inicializar
um \textit{buffer} de 128 \textit{bits} com um valor fixo. 

Em cada rodada, um bloco de entrada de 512 \textit{bits} é extraído e colocado no \textit{buffer} de 128 \textit{bits}. Para que os cálculos sejam feitos com maior precisão, também é incluída uma tabela
criada a partir da função de seno. Há quatro rodadas para cada bloco de entrada. Esse processo continua até que todos os blocos de entrada tenham sido consumidos. O conteúdo do
\textit{buffer} de 128 \textit{bits} forma o sumário de mensagens.

O MD5 já é usado há mais de uma década, e muitos pessoas o têm atacado. Foram encontradas algumas vulnerabilidades, mas certas etapas internas impedem que ele seja rompido. Porém,
se as barreiras restantes dentro do MD5 caírem, ele poderá falhar eventualmente. Apesar disso, atualmente, ele ainda resiste.


\subsection{SHA-1}

A outra função importante do sumário de mensagens é o SHA-1 (\textit{Secure Hash Algorithm 1}), desenvolvido pela NSA e aprovado pelo NIST no FIPS 180-1. A exemplo do MD5, esse
algoritmo processa os dados de entrada em blocos de 512 \textit{bits}. No entanto, ele gera um sumário de mensagens de 160 \textit{bits}.

O algoritmo do SHA-1 começa preenchendo a mensagem com a adição de um \textit{bit} 1 ao final, seguido pelo número de \textit{bits} 0 necessários para tornar o tamanho um múltiplo
de 512 \textit{bits}. Em seguida, um número de 64 \textit{bits} contendo o tamanho da mensagem antes do preenchimento é submetido a uma operação OR nos 64 \textit{bits} de baixa
ordem. Com os computadores, essa orientação corresponde a máquinas bi g-endian, como SPARC, mas o SHA-1 sempre preenche o fim da mensagem, não importando que máquina endian seja usada.

Durante a computação, o SHA-1 mantém cinco variáveis de 32 \textit{bits}, de $H_0$ a $H_4$, onde o \textit{hash} se acumula. Elas são inicializadas como constantes especificadas no
padrão. Cada um dos blocos $M_0$ a $M_{n-1}$ é agora processado. Para o bloco atual, primeiro as 16 palavras são copiadas no início de um \textit{array} auxiliar de 80 palavras, $W$.
Em seguida, as outras 64 palavras em $W$ são preenchidas, usando-se a fórmula:

$W_i$ $=$ $S^1$ ($W_{i-3}$ XOR $W_{i-8}$ XOR $W_{i-14}$ XOR $W_{i-16}$ ), (16 $\le$ i $\le$ 79)


onde $S^b$ ($W$) representa a rotação circular à esquerda da palavra de 32 \textit{bits}, $W$, por $b$ \textit{bits}. Agora, são inicializados cinco variáveis de rascunho de $A$ até $E$,
com valores de $H_0$ a $H_4$, respectivamente.


Agora que o primeiro bloco de 512 \textit{bits} foi processado, o próximo é iniciado. O \textit{array} $W$ é reinicializado a partir do novo bloco, mas $H$ fica como estava. Quando esse
bloco é concluído, o próximo é iniciado e assim por diante, até todos os blocos da mensagem de 512 \textit{bits} serem processados. Quando o último bloco é concluído, as cinco
palavras de 32 \textit{bits} no \textit{array} $H$ são transmitidas como saída, formando o \textit{hash} criptográfico de 160 \textit{bits}. O código C completo para SHA-1 é dado na
RFC 3174\footnote{http://www.apps.ietf.org/rfc/rfc3174.html}.

Novas versões de SHA-1 estão em desenvolvimento para hashes de 256, 384 e 512 \textit{bits}, respectivamente.



\section{Considerações Finais}

A segurança na Tecnologia da Informação, especialmente na área de Redes de Computadores, é abordada como um dos principais temas da Computação, desde a Literatura mais
antiga até a atual. Muitos profissionais não medem esforços nesses estudos, a fim de obter os melhores resultados e soluções.

Tudo isso, aliado ao crescimento vertiginoso das tecnologias, o que, inevitavelmente, gera brechas de segurança pelo caminho, garante que nunca faltarão temas para estudo
na área de Segurança, de forma que um pesquisador ou profissional desse ramo jamais ficará ocioso.



\bibliography{texto}



\end{document}

































