\documentclass{article}
\usepackage[utf8]{inputenc}
\usepackage[brazil]{babel}
\usepackage{hyperref}
\usepackage[paper=a4paper,lmargin=1.5cm, rmargin=1.5cm, tmargin=1.6cm, bmargin=1.5cm]{geometry}


\title{\textit{System Calls} - Chamadas ao Sistema}
\author{Roberto Beraldo Chaiben - GRR 20084213}
\date{30/06/2010}


\begin{document}


\maketitle


\section{Definição}

Uma \textit{System Call}, também chamada apenas de \textit{syscall}, consiste em uma função que é chamada pelo programador, a fim
de executar tarefas do Sistema Operacional. Quando uma \textit{syscall} é chamada, o processo corrente é interrompído, temporariamente,
passando a execução ao Sistema Operacional, que executa a operação solicitada no modo supervisor (administrador).


\subsection{POSIX}


\begin{quotation}

POSIX (acrônimo para \textit{Portable Operating System Interface}, que pode ser traduzido como Interface Portável entre Sistemas Operacionais)
é uma família de normas definidas pelo IEEE e designada formalmente por IEEE 1003, que tem como objetivo garantir a portabilidade do código-fonte
de um programa a partir de um sistema operacional que atenda às normas POSIX para outro sistema POSIX. Desta forma as regras atuam como uma
interface entre sistemas operacionais distintos. A designação internacional da norma é ISO/IEC 9945.

\end{quotation}

\begin{flushright}
Fonte: http://pt.wikipedia.org/wiki/POSIX
\end{flushright}


As \textit{syscalls} utilizadas em Linux seguem o padrão POSIX.


Em Linux, devido à existência do usuário \textit{root}, que detém acesso total ao sistema, somente ele pode realizar determinadas tarefas,
como acesso a dispositivos de I/O (\textit{Input/Output} - Entrada/Saída). Logo, quando um processo executado em modo usuário comum precisa
realizar esse tipo de tarefas, ele precisa solicitar essa ação ao superrvisor (Sistema Operacional). Para isso são utilizadas \textit{syscalls}.
Ao chamar uma \textit{syscall}, o processo é interrompido e a execução passa ao modo \textit{root}, que executa a solicitação. Feito isso, o
processo volta a ser executado normalmente, em modo usuário comum.


Em assembly intel x86, podemos chamar uma \textit{syscall} da seguinte forma:

\begin{verbatim}

movl SYSCALL_NUMBER, %eax
int 0x80

\end{verbatim}

\textbf{SYSCALL\_NUMBER} deve ser substituído pelo número da \textit{syscall} que se deseja chamar. O número da syscall é sempre movido
ao registrador EAX. Os demais registradores recebem os parâmetros da syscall.



\subsection{Win32}

O funcionamento das syscalls em sistemas Windows é semelhante ao o que ocorre no padrão POSIX. 

Para se chamar uma syscall win32 em assembly, usa-se o código abaixo:

\begin{verbatim}
movl SYSCALL_NUNBER, %eax
int 0x2e
\end{verbatim}

Como se pode notar, a única diferença é o número da instrução, que passou de 0x80 para 0x2e.




\section{Como Inserir uma Syscall no Kernel do Linux, na plataforma x86}


\subsection{Versões das Ferramentas}

Para a elaboração deste texto, usou-se o Kernel 2.6.32, obtido em http://www.kernel.org, e
a distribuição Ubuntu, versão 10.04, 32 bits, rodando em uma máquina virtual do Virtual Box.


\subsection{Arquivos Modificados}

Assumiremos que os caminhos aqui presentes são relativos ao diretório raiz da distribuição do Kernel baixada e descomprimida
a partir do site oficial, ou seja, o diretório \textbf{linux-2.6.32}, neste caso.

Também assumiremos que o nome da syscall criada será \textbf{mysyscall}.

\begin{itemize}

	\item arch/x86/kernel/syscall\_table\_32.S
	\item include/asm/unistd\_32.h
	\item arch/x86/include/asm/syscalls.h
	\item Makefile

\end{itemize}


\subsection{Diretórios e Arquivos Criados}

É preciso criar um diretório na raiz do diretório do Kernel, que conterá o arquivo fonte e o Makefile para a nova syscall.
Supondo que a syscall se chamará \textbf{mysyscall}, criaremos o diretório \textbf{mysyscall}. Nele, criaremos os seguintes
arquivos (caminhos a partir da raiz do Kernel):

\begin{itemize}

	\item mysyscall/mysyscall.c
	\item mysyscall/Makefile

\end{itemize}


\subsection{Modificações em arch/x86/kernel/syscall\_table\_32.S}

Aofinal desse arquivo deve ser inserida a seguinte linha:

\begin{verbatim}
.long sys_mysyscall
\end{verbatim}


\subsection{Modificações em include/asm/unistd\_32.h}

Toda syscall possui um número (ID único). Esse arquivo relaciona a nova syscall a um número. Deve-se verificar o último número
utilizado e somarmos uma unidade para termos o ID da nova syscall. Supondo que o último número utilizado seja 1079, a linha a
ser inserida no final do arquivo será:

\begin{verbatim}
#define __NR_mysyscall 1080
\end{verbatim}


\subsection{Modificações em arch/x86/include/asm/syscalls.h}

Esse arquivo contém os cabeçalhos de todas syscalls. Vamos adicionar a seguinte linha no final do arquivo:


\begin{verbatim}
asmlinkage long sys_mysyscall(void);
\end{verbatim}

O identificador \textit{asmlinkage} deve estar presente no início de todos os protótipos de syscalls. Estamos
considerando uma syscall sem parâmetros, por isso a palavra \textit{void}.


\subsection{Modificações no Makefile}

Adicionar na linha onde há \textbf{core-y +=} (653 no Makefile utilizado) o nome do diretório onde ficará o fonte da syscall
(mysyscall, nesse caso).


\subsection{Conteúdo de mysyscall/mysyscall.c}

Nesse arquivo ficará o código-fonte da nova syscall. 

Toda implementação de syscall deve incluir "linux/linkage.h".


\begin{verbatim}
#include "linux/linkage"

asmlinkage int sys_mysyscall(void)
{
    // implementação da syscall
}
\end{verbatim}


\subsection{Código de mysyscall/Makefile}

O Makefile para a nova syscall conterá apenas isto:


\begin{verbatim}
obj-y := mysyscall.o 
\end{verbatim}


\begin{verbatim}

\end{verbatim}










\section{Referências}

\begin{itemize}

	\item Criação de System Calls em Linux e Compilação de Kernel:
	
	\begin{itemize}
	
		\item http://tldp.org/HOWTO/html\_single/Implement-Sys-Call-Linux-2.6-i386/
		\item http://www.csee.umbc.edu/courses/undergraduate/CMSC421/fall02/burt/projects/howto\_add\_systemcall.html
		\item http://courses.cs.vt.edu/\~cs3204/spring2004/Projects/2/InstructionsToAddSystemCallToLinux\_rev00.pdf
		\item http://gerencievocemesmo.com.br/site/?p=281
		\item http://www.falkotimme.com/howtos/debian\_kernel2.6\_compile/
	
	\end{itemize}
	
	
	\item Definições de System Calls:
	
	\begin{itemize}
	
		\item http://www.softpanorama.org/Internals/unix\_system\_calls.shtml
		\item http://www.symantec.com/connect/pt-br/articles/windows-syscall-shellcode
		\item http://msdn.microsoft.com/pt-br/library/aa908825.aspx
		\item http://www.codeguru.com/Cpp/W-P/system/devicedriverdevelopment/article.php/c8035/
	
	\end{itemize}

\end{itemize}





\end{document}














