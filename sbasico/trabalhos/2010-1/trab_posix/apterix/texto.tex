\documentclass[12pt]{article}

\usepackage[utf8]{inputenc}
\usepackage[brazil]{babel}
\usepackage{verbatim} 
\usepackage{listings}

\begin{document}

\setlength{\parskip}{2ex}

\begin{tabular}[l]{ | p{10cm} | l | }
	\hline \multicolumn{2}{|c|}{Trabalho de Software Básico - Turma 2010-1} \\
	\hline Nome: Francisco Panis Kaseker &
	GRR20071909 \\
	\hline Título: Explicação e implementação de uma SYSCALL &
	Data: 30/06/2010 \\
	\hline
\end{tabular}


\vspace{2cm}

\section{Introdução}

Basicamente uma \textit{SYSCALL} é uma chamada da aplicação feito ao \textit{Kernel} do
sistema operacional. Essa chamada executa alguma rotina no sistema, mas dentro
da camada do \textit{Kernel}. A segunda seção explicará com maior profundidade do
que tratam as \textit{SYSCALLs} (história, exemplos etc) e, por fim, a terceira
tratará de explicar como adicionar uma chamada ao sistema no Linux.

\section{SYSCALLs: O que são}

\subsection{História}
Devido a necessidade de alguns aplicativos precisarem obter determinados ações
ou resultados do sistema operacional sem para isso terem de rodar sob os
privilégios do modo \textit{root}, as \textit{SYSCALLs} vieram para suprir essa
necessidade sem para isso diminuir a segurança do sistema operacional.

\subsection{Porque usá-las}
Essencialmente para garantir segurança ao sistema ao não permitir que os
aplicativos tenham que rodar sob os privilégios do \textit{root}, mas que possam
desempenhar suas funções no sistema normalmente.

\subsection{Exemplos}
Nos sistemas \textit{UNIX-like} nós temos a \textit{POSIX} que desempanhm diversas funções (como,
por exemplo, open, read, write, close, wait, exec, fork, exit kill etc). Nos
sistemas baseados em Windows, temos a \textit{WIN32}, somente implementadas pela
Microsoft. Parte delas está descrita na documentação da Microsoft e outra parte
não.

\subsection{Implementaçõs mais comuns} 
As chamadas ao sistema possuem diversas utilidades atualmente. Na parte de
controle de processos, por exemplo, temos a criação ou fechamento de um
processo ou alocação e liberação de memória. Para o gerenciamento de arquivos,
temos a abertura, fechamento, escrita ou leitura de arquivos. Há ainda chamadas
ao sistema para envio de dados na rede, gerenciamento de drivers (hd, cd, dvd
etc) e tantas outras funções cujo controle é somente do sistema operacional ou
do \textit{root} .

\subsection{O modelo POSIX}
O modelo \textit{POSIX} é uma padronização de chamadas do sistema para tentar garantir a
portabilidade de um código em diferentes sistemas operacionais. O modelo \textit{POSIX} é
seguido pelos sistemas Unix-like, porém isso não é uma regra. O sistema \textit{FreeBSD},
por exemplo, tem mais que o dobro de \textit{SYSCALLs} que o definido pela \textit{POSIX}.

\subsection{O modelo WIN32}
Essa API é desenvolvida para o sistema operacional Windows, da Microsoft. Ela
abrange desde os sistemas 16bits até os 64bits do seu portfólio.
Embora a WIN32 seja um modelo para fornecer recursos do sistema operacional aos
aplicativos, ela permite que outras API's tenham acesso diretamente aos recursos
de hardware do sistema, especialmente para hardware gráfico e de rede.

\section{Implementando uma SYSCALL}
Será implementado uma \textit{SYSCALL} no \textit{Kernel} Linux x86 que tem a função de
exibir quantas vezes ela foi chamada. Será usado uma variável global como
contador e a \textit{SYSCALL} terá de somar o valor um a essa variável global sempre
que for chamada. Todo printk ("printf do Kernel") será ligo no seguinte log: /var/log/syslog\\
A \textit{SYSCALL} aqui implementada foi a de número 339.

\subsection{Compilando o Kernel}

Foi utilizado o tutorial abaixo para compilar o Kernel; Importante: O autor do
artigo do site abaixo referenciado é o mesmo autor desse trabalho.
http://gerencievocemesmo.com.br/site/?p=281

\subsection{Implementando uma SYSCALL no Linux x86}

\begin{enumerate}

\item Baixar o Kernel e extraí-lo em algum lugar; Usei o "/usr/src/".

\item Para facilitar, criei um link simbólico para usar o path "/usr/src/linux"
como source do Kernel.

\item Crie um arquivo chamado "minhasys.c" em "/usr/src/kernel/minhasys/". O seu conteúdo
será o seguinte:

%code
\begin{lstlisting}
#include <linux/linkage.h>
asmlinkage int sys_minhasys() {
MINHAVARIAVELGLOBAL++;
printk("CHAMADA %d VEZES!\n", MINHAVARIAVELGLOBAL);
return 0;
}
\end{lstlisting}

\item Agora vamos modificar alguns arquivos para tornar a nova \textit{SYSCALL}
disponível ao \textit{Kernel}.

\begin{itemize}
\item arquivo: /usr/src/linux/arch/x86/include/asm/unistd\_32.h \\
adicione na ultima linha: \#define \_\_NR\_minhasys         NUMERO\\
Substitua NUMERO pelo último número do define da lista incrementado de "1".

\item arquivo: /usr/src/linux/arch/x86/include/asm/syscalls.h\\
adicione ao fim da lista: asmlinkage int sys\_helloworld();

\item arquivo: /usr/src/linux/arch/x86/kernel/syscall\_table\_32.c\\
adicione: .long sys\_minhasys

\item arquivo: /usr/src/linux/Makefile
linha original: core-y += kernel/ mm/ fs/ ipc/ security/ crypto/ block/\\
nova linha: core-y += kernel/ mm/ fs/ ipc/ security/ crypto/ block/ minhasys/

\end{itemize}

\item Agora crie um arquivo chamado "Makefile" dentro do diretório "minhasys"
com o seguinte conteúdoo:\\
obj-y := minhasys.o

\item O contador (variável int MINHAVARIAVELGLOBAL) foi declarado dentro da syscall como static, assim a declaração será feita uma única vez e o valor não será perdido.

\item Agora é só compilar o \textit{Kernel}.

\end{enumerate}

\section{Dicas adicionais}

\begin{itemize}

	\item O NUMERO de uma \textit{SYSCALL} pode ser qualquer uma dentro do formato (int), desde que já não esteja sendo usada.
	\item No how-to acima a syscall foi colocada dentro de um arquivo separado, mas poderia ter sido incluída facilmente dentro do arquivo "kernel/sys.c", junto as outras \textit{SYSCALL}; Fazendo isso, não é necessário criar ou modificar nenhum Makefile.

\end{itemize}

\section{Referências}

\begin{itemize}

	\item http://macboypro.wordpress.com/2009/05/15/adding-a-custom-system-call-to-the-linux-os/
	\item http://gerencievocemesmo.com.br/site/?p=281
	\item http://www.linuxchix.org/content/courses/kernel\_hacking/lesson8
	\item http://ubuntuforums.org/attachment.php?attachmentid=149947\&d=1268472461
	\item http://en.wikipedia.org/wiki/System\_call
	\item http://pt.wikipedia.org/wiki/POSIX
	\item http://en.wikipedia.org/wiki/Windows\_API
	\item http://blog.veiga.eti.br/niveis-de-log-do-printk-para-depurar-o-kernel-pela-console/

\end{itemize}

\end{document}
