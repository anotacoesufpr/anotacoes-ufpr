\documentclass{article}
\usepackage[utf8]{inputenc}
\usepackage[brazil]{babel}
\usepackage{hyperref}
\usepackage[paper=a4paper,lmargin=2.0cm, rmargin=1.50cm, tmargin=1.00cm, bmargin=1.50cm]{geometry}


\title{Trabalho 1 de Redes de Computadores I}
\author{Roberto Beraldo Chaiben  \\ Carlos Augusto Ligmanowski Júnior }
\date{25/06/2010}


\begin{document}


\maketitle


\section{Autores}


\begin{itemize}

	\item Roberto Beraldo Chaiben - GRR 20084213 - login: rbc08
	
	\item Carlos Augusto Ligmanowski Júnior - GRR 20081238 - login: calj08

\end{itemize}


\section{Comandos aceitos pelo Programa}


\begin{itemize}

	\item \textbf{l}: faz um ls no diretório corrente
	
	\item \textbf{c dir}: faz cd para dir
	
	\item \textbf{f arquivo}: exibe o conteúdo de arquivo
	
	\item \textbf{a arquivo}: faz um append em arquivo
	
	\item \textbf{w arquivo}: Edita uma linha de arquivo
		
	\item \textbf{q}: Sai do programa 
	
\end{itemize}


\section{Estrutura de Arquivos e Diretórios}

\begin{itemize}

	\item \textbf{bin}: Diretório onde são gerados os executáveis \textbf{cliente} e \textbf{servidor}
	
	\item \textbf{Makefile}: Makefile do trabalho
	
	\item \textbf{src}: Diretório com os arquivos fontes C++
	
	\item \textbf{src/cliente.cpp}: Arquivo com o \textbf{main} para o cliente
	
	\item \textbf{src/servidor.cpp}: Arquivo com o \textbf{main} para o servidor
	
	\item \textbf{src/lib}: Diretório com os arquivos fontes C++ das classes do trabalho (bibliotecas)

\end{itemize}



\section{Procedimentos para Execução}


Dentro do diretório raiz do trabalho, digite:

\vspace{0.5cm}

\$ make

\vspace{0.4cm}

Esse comando gera os arquivos \textbf{bin/cliente} e \textbf{bin/servidor}. Depois basta executá-los, como root.








\end{document}
