\documentclass{article}


\usepackage[utf8]{inputenc}
\usepackage[T1]{fontenc}
\usepackage[brazil]{babel}
\usepackage[pdftex,pagebackref=true,colorlinks=true,linkcolor=blue,unicode]{hyperref}
\usepackage[paper=a4paper,lmargin=2.0cm, rmargin=2.0cm, tmargin=2.0cm, bmargin=2.0cm]{geometry}
\usepackage{graphicx}
\usepackage{makeidx}
\usepackage{hyperref}
\usepackage{enumerate}
\usepackage{indentfirst}


\title{Apostila de Redes Móveis - UFPR}
%\author{}
\date{2011}


\makeindex


\begin{document}


\maketitle
\tableofcontents


\newpage


%====================================================

\section{Telefonia Móvel}


\subsection{Conceitos Iniciais}

\begin{itemize}
	\item Estação Móvel / Celular
	\item Antena
	\item Células
	\item Estação Rádio Base (ERB)
	\item Central de Comutação Móvel (\textit{Mobile Switching Center - MSC})
	\item Canal Direto (\textit{Forward Channel})
	\item Canal Reverso (\textit{Reverse Channel})
	\item Canal de Controle
\end{itemize}


\subsection{AMPS}

O AMPS foi o primeiro sistema de telefonia celular, inicialmente usado em
Chicago, em 1963.\\

Células grandes: 3400 km$^2$

\subsection{ETACS}


%====================================



\printindex



\end{document}
