\documentclass{book}

\usepackage[utf8]{inputenc}
\usepackage[brazil]{babel}
\usepackage[pdftex,pagebackref=true,colorlinks=true,linkcolor=blue,unicode]{hyperref}
\usepackage[paper=a4paper,lmargin=2.0cm, rmargin=2.0cm, tmargin=2.0cm, bmargin=2.0cm]{geometry}
\usepackage{graphicx}
\usepackage{makeidx}

\title{Trabalho de Análise e Projeto de Sistemas}
\author{Carlos Augusto Ligmanowski \\
		Francisco Panis Kaseker \\
		Leandro José Machado Vargas \\
		Roberto Beraldo Chaiben
}

\date{2010-2}

\makeindex

\begin{document}


\maketitle
\tableofcontents


%====================================================


\chapter{Introdução}

\section{Sobre este documento}

Trabalho desenvolvido para a disciplina de Análise e Projeto de Sistemas, Profª. Laura Sánchez García, no segundo semestre de 2010. \\


O trabalho consiste na análise do sistema eletrônico de gestão da graduação do Departamento de Informática\footnote{http://www.inf.ufpr.br} da Universidade Federal do Paraná.


\section{Sobre os autores}

Os autores deste trabalho são alunos do curso de Bacharelado em Ciência de Computação\footnote{http://www.inf.ufpr.br/bcc}, na Universidade Federal do Paraná\footnote{http://www.ufpr.br}.\\

\begin{itemize}
	\item Carlos Augusto Ligmanowski - GRR20081238
	\item Francisco Panis Kaseker - GRR20071909
	\item Leandro José Machado Vargas - GRR20072734
	\item Roberto Beraldo Chaiben - GRR20084213
\end{itemize}


%===================================================

\chapter{Levantamento e Análise de Requisitos}


A etapa de levantamento de requisitos é composta por diversas técnicas que visam a obter do cliente as
informações necessárias para desenvolver o projeto do sistema de informação. Dentre essas técnicas, temos:\\

 
\begin{itemize}
	\item \textbf{Entrevistas}: entrevistas com clientes e usuários potenciais;
	\item \textbf{Observação}: observação do comportamento dos usuários em seu ambiente de trabalho;
	\item \textbf{\textit{Brainstorming}}: reunião com várias pessoas, onde todos discutem um tema central;
	\item \textbf{Análise de textos}: o usuário descreve as necessidades textualmente;
	\item \textbf{Reutilização de requisitos}: Reaproveitamento de padrões ou requisitos de outros sistemas. 
\end{itemize}


Cada uma das seguintes seções abordarão os seguintes perfis individualmente:

\begin{enumerate}
	\item \textbf{Perfil dos Professores}: detalhes relacionados à interface dos professores e às funcionalidades
	que ela deve possuir;
	\item \textbf{Perfil da Coordenação}: detalhes relacionados à interface da coordenação e às funcionalidades
	que ela deve possuir;
	\item \textbf{Perfil dos Alunos}: detalhes relacionados à interface dos alunos e às funcionalidades
	que ela deve possuir;
	\item \textbf{Perfil da Secretaria}: detalhes relacionados à interface da secretaria e às funcionalidades
	que ela deve possuir;
\end{enumerate}


Quando não houver necessidade de dividir dados nos perfis acima, ou existir dados comuns a todos eles, será citado
em um \textbf{Perfil Geral}.


\section{Requisitos Funcionais}

Requisitos funcionais são aqueles intimamente ligados ao sistema, essenciais para ele, como documentos
específicos, campos de formulário, ações de um menu de opções etc.

\subsection{Perfil de Professores}

\begin{itemize}
	\item Atualização de dados cadastrais, como nome, endereço, telefone, CPF, RG etc
	\item \textbf{Gerenciamento de frequência dos alunos}: emissão de lista de presença, controle de faltas dos alunos;
	\item \textbf{Acompanhamento de notas}: cadastro de notas dos alunos, com operações de adição, remoção, atualização e visualização;
	\item \textbf{Consulta ao ensalamentos das disciplinas que cada professor ministra}: consulta ao local onde ministrará aulas de suas disciplinas;
\end{itemize}




\subsection{Perfil da Coordenação}

\begin{itemize}
	\item \textbf{Gerenciamento de ensalamento das disciplinas}: definir os ensalamentos das turmas de cada disciplina;
	\item \textbf{Gerenciamento de matrículas do curso}
		\begin{itemize}
			\item Gerenciamento de matrículas em disciplinas isoladas
			\item Quebra de pré-requisitos
			\item Validação de disciplinas por equivalência
			\item Fixar a lista de oferta de disciplinas
			\item Correção de Matrícula
			\item Cancelamento de matrícula: somente até antes de decorrida a metade do período letivo
		\end{itemize}
	\item \textbf{Trancamento e destrancamento de curso}
	\item \textbf{Cancelamento de registro acadêmico}
	\item \textbf{Efetivação de registro acadêmico}
	\item \textbf{Requisição de transferência de curso}
	\item \textbf{Criação de período especial}: criação de turmas em períodos especiais, como durante as férias;% (art. 86, 37/97)
	\item \textbf{Envio de lista ao DAA de possíveis formandos}% (art. 110, 37/97)
	\item \textbf{Geração de certificado de aprovação}% (art. 112, 37/97)
	\item \textbf{Geração de comprovante de aprovação (válido até a chegada do certificado de conclusão)}
\end{itemize}



\subsection{Perfil de Alunos}

\begin{itemize}
	\item Atualização de dados cadastrais, como nome, endereço, telefone, CPF, RG etc
	\item Solicitação de matrículas;
	\item Consulta ao histórico;
	\item Comprovante de matrículas;
	\item Cancelamento de matrículas;
	\item Trancamento de curso;
	\item Solicitação de prova de adiantamento de disciplina;
	\item Solicitação de exercícios domiciliares% (art. 85, 37/97);
	\item Solicitação de revisão de prova;
	\item Solicitação de segunda chamada de prova.
\end{itemize}




\subsection{Perfil da Secretaria}

\begin{itemize}
	\item Divulgação de editais;
	\item Gerenciamento de faltas dos docentes;
	\item Impressão de documentos;
	\item Orientações gerais.
\end{itemize}




%===============================================================




\section{Requisitos não-funcionais}

Requisitos não-funcionais estão relacionados com o ambiente em que o sistema será executado. Isso
envolve detalhes de segurança, tecnologias, \textit{hardware} etc.\\

Possíveis subdivisões dos requisitos não-funcionais podem ser: \\

\begin{itemize}
	\item Requisitos operacionais;
	\item Requisitos de segurança;
	\item Requisitos de desempenho;
	\item Especificações de Hardware e software.
\end{itemize}


\subsection{Perfil Geral}

\begin{itemize}
	\item \textbf{Estabilidade}: o sistema deve suportar o tráfego na maioria dos casos,
	especialmente em períodos críticos, como o de matrículas, em que muitos usuários utilizam o 
	sistema simultaneamente;
	\item \textbf{Interoperabilidade}: capacidade de se integrar e se comunicar com sistemas semelhantes,
	por meio de uso de padrões abertos, com bases de dados em XML
	\item \textbf{Confiabilidade}: o sistema deve ser um reflexo fiel da realidade, sem
	discrepância de informações e sempre atualizado;
	\item \textbf{Desempenho}: o sistema deve ser rápido, não podendo levar mais de 30 (trinta)
	segundos para processar a informação solicitada;
	\item \textbf{Escalabilidade} O sistema deve ser modularizado o suficiente para permitir a expansão do sistema e suas funções;
	\item \textbf{Usabilidade}: o sistema deve ser fácil de ser utilizado por todos os usuários,
	desde os leigos até os experientes;
	\item \textbf{Acessibilidade}: o sistema deve prover acessibilidade para portadores de deficiências,
	permitindo navegação inteiramente via teclado, possibilidade de aumento de fontes de textos, links orientados aos programas de leitura de texto, textos que substituam imagens etc;
\end{itemize}


%======================================================


\section{Requisitos Normativos}

Requisitos normativos, ou restritivos, são informações como custos, prazos, plataforma tecnológica, aspectos legais, comunicação com outros sistemas e regras de negócio;


\subsection{Perfil Geral}

\begin{itemize}
	
	\item Só são permitidos acessos ao sistema pelos usuários vinculados ao Departamento de Informática da UFPR;
	\item Adequação às leis do regimento interno da Universidade e da legislação do
MEC;
	\item Plataforma tecnológica: software livre e gratuito.
\end{itemize}


\subsection{Perfil de Professores}

\begin{itemize}
	\item Cada professor só pode acessar e modificar informações referentes às disciplinas que ministra
\end{itemize}


\subsection{Perfil da Coordenação}

\begin{itemize}
\item A Coordenação não pode modificar dados como notas e frequências de alunos, apenas consultá-las, visto que são atribuições restritas aos professores
\end{itemize}



\subsection{Perfil de Alunos}

\begin{itemize}
	\item Caso o aluno tenha quatro (04) ou mais reprovações pendentes em disciplinas obrigatórias distintas, ele
deverá matricular-se somente nestas disciplinas;
	\item Máximo de 3 (três) disciplinas isoladas simultâneas por aluno por semestre letivo;
	\item A carga horária não pode ultrapassar 8 (oito) horas diárias ou 40 (quarenta) horas semanais de aula;
	\item Não deve permitir que o aluno solicite uma matéria como eletiva em outro curso se essa for uma matéria obrigatória do seu currículo.
\end{itemize}



%================================================================


\section{Requisitos Inovadores}
São os requisitos que aperfeiçoam a interface entre o usuário e o sistema, melhorando a usabilidade e, portanto, a eficiência do trabalho; Também pode colaborar com a diminuição da taxa de erros e o controle dos erros.

\subsection{Perfil Geral}

\begin{itemize}
	\item Envio de SMS com informações gerais para alunos;
	\item Acesso por meio de dispositivos móveis;
	\item Autenticação por meio de chaves SSH.
\end{itemize}



\subsection{Perfil dos Alunos}

\begin{itemize}
	\item Avaliação de professores ao fim de cada semestre;
	\item Sistema de fila de espera para matrículas.
\end{itemize}


\subsection{Perfil dos Professores}

\begin{itemize}
	\item \textbf{Gerenciamento de trabalhos}: criação de trabalhos práticos de disciplinas, controle de entrega e sistema de envio de mensagens sobre os trabalhos;
	\item \textbf{Solicitação de criação de disciplina à Coordenação}: conforme achar conveniente, um professor pode solicitar à coordenação a criação de uma
	disciplina. O módulo deve possibilitar informar detalhes como possível nome para a disciplina, descrição, ementa e objetivos.
\end{itemize}



\subsection{Perfil da Secretaria}

\begin{itemize}
	\item \textbf{Lista de Possíveis Formandos}: possibilitar geração de lista de possíveis formandos, com base nas disciplinas que os alunos estão cursando no semestre corrente.
\end{itemize}




%====================================

\printindex


\end{document}

