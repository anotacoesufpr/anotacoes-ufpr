\documentclass{article}


\usepackage[utf8]{inputenc}
\usepackage[T1]{fontenc}
\usepackage[brazil]{babel}
\usepackage[pdftex,pagebackref=true,colorlinks=true,linkcolor=blue,unicode]{hyperref}
\usepackage[paper=a4paper,lmargin=2.5cm, rmargin=2.5cm, tmargin=3.0cm, bmargin=3.0cm]{geometry}
\usepackage{graphicx}
\usepackage{makeidx}
\usepackage{hyperref}
\usepackage{enumerate}
\usepackage{indentfirst}


\title{}
\author{}
%\date{}


\makeindex


\begin{document}


%\maketitle
%\tableofcontents

%\newpage


%====================================================

\setlength{\parskip}{2ex}

% cabeçalho
\begin{tabular}[l]{ | p{10cm} | l | }
  \hline
  \multicolumn{2}{|c|}{CI205 - Administração da Produção de Informática - Turma 2011-1} \\
  \hline
  Nome: Roberto Beraldo Chaiben & 
  GRR 20084213 \\
  \hline
  Título: Três inovações em processo de produção pessoal profissional &
  Data: 05/04/2011 \\
  \hline
\end{tabular}


\vspace{2cm}


\section*{Introdução}

O processo de produção profissional e pessoal necessita de constantes inovações
e mudanças, a fim de encontrar o equilíbrio ideal para o funcionamento adequado
do sistema, seja ele uma empresa ou um profissional autônomo.

Há diversos recursos para serem usados em inovações. Alguns mais importantes e
outros menos, conforme a área de atuação da empresa ou do profissional.


\section{Otimização de Tempo e Qualidade}

Otimizar o tempo e a qualidade do produto ou serviço de um empresa ou profissional
é um dos pontos principais da Administração. Excelência na qualidade e agilidade
na produção resultam em margens de lucro maiores.

\subsection{Lei de Pareto}

A \textbf{Lei de Pareto} afirma que 80\% da conseguências advêm de 20\% das causas.
Transpondo essa regra para a Administração da Produção da Informática, podemos
afirmar que 80\% do lucro de uma empresa provém de 20\% do trabalho.

Assumindo a veracidade dessa regra, não compensa investir recursos e tempo em
trabalhos que não renderão bons resultados à instituição. É mais produtivo
centrar esforços nos 20\% de trabalho que mais proporcionam resultados positivos
para o empreendimento.


\subsection{Divisão de tempo}

A divisão adequada do tempo é outro fator decisivo para a boa produtividade de
uma instituição. Há três ações que merecem destaque dentre as realizadas
por empresas:

\begin{enumerate}
	\item \textbf{Inovações}: novos recursos e funcionalidades para aprimorar o
	produto ou serviço prestado pela empresa;
	\item \textbf{Melhorias}: aperfeiçoamento dos recursos e funcionalidades do
	produto ou serviço atualmente fornecido pela empresa;
	\item \textbf{Manutenção}: produção do produto ou serviço oferecido pela empresa;
\end{enumerate}


No que diz respeito à Administração da Produção da Informática, deve-se dedicar
mais tempo às \textbf{melhorias}. A \textbf{manutenção} é o foco principal da
equipe técnica, e as \textbf{inovações} dizem respeito à Administração Geral do
empreendimento.


\subsection{Redução de custo e tempo de execução}

Reduzir o tempo de execução das tarefas permite à empresa finalizar e entregar
seu produto ou serviço mais rapidamente. Com isso, pode-se realizar outros
trabalhos mais cedo, resultando em novos clientes e mais lucro.

A isso está ligada  redução de custos para a produção do produto. Essas duas
práticas aliadas resultam em alta produtividade e alto lucro.

A fim de alcançar esse ideal, pode-se fazer uso de ferramentas e rotinas já
existentes e funcionais. Na produção de \textit{software}, por exemplo, o
reaproveitamento de código e a abstração fornecidos pela Orientação a Objetos
são pontos fundamentais.


\subsection{Extrair o máximo de informações do cliente}

Para que a empresa desenvolva a solução ideal para seu cliente, é preciso que
ela conheça e compreenda integralmente a necessidade do consumidor. Desenvolver
uma solução que não atende exatamente à necessidade do cliente caracteriza perda
de tempo, uma vez que a equipe técnica terá de fazer manutenção do produto,
adequando-o às especificações passadas pelo cliente.

Por isso é extremamente necessário consultar o cliente e obter o maior número de
informações sobre os requisitos do sistema de que ele necessita. Sabendo
exatamente o que deve ser implementado, a equipe técnica produzirá a solução
adequada ao consumidor.





\section{\textit{Mura, Muri, Muda}}

As palavras \textit{Mura, Muri} e \textit{Muda} são palavras japonesas que
significam, respectivamente, \textbf{ausência de regularidade}, \textbf{sobrecarga}
e \textbf{desperdício}. Ou seja, são características que devem ser \textbf{evitadas}
em qualquer situação.

É necessário haver \textbf{regularidade}, a fim de disciplinar os funcionários,
que devem estar cientes de seus encargos e dos prazos para sua execução.

Porém, não se deve sobrecarregar os funcionários, evitando estresse e conflitos.
O excesso de trabalho prejudica a eficiência do profissional.

O desperdício deve ser sempre evitado, em qualquer situação. Todo desperdício
causa prejuízo.







\section{Cinco Esses - 5S}

Os Cinco Esses (5S) são técnicas, que devem ser preferencialmente, desde o
início do empreendimento, as quais visam à organização, limpeza e padronização
de processos.

Os 5 Ss são:

\begin{enumerate}
	\item \textit{Seiri}: \textbf{Senso de utilização}: Refere-se à prática de
	verificar todas as ferramentas, materiais, etc. na área de trabalho e manter
	somente os itens essenciais para o trabalho que está sendo realizado. Tudo
	mais é guardado ou descartado. Este processo conduz a uma diminuição dos
	obstáculos à produtividade do trabalho.
	
	\item \textit{Seiton}: \textbf{Senso de ordenação}. Enfoca a necessidade de
	um espaço organizado. A organização, neste sentido, refere-se à disposição
	das ferramentas e equipamentos em uma ordem que permita o fluxo do trabalho.
	Ferramentas e equipamentos deverão ser deixados nos lugares onde serão
	posteriormente usados. O processo deve ser feito de forma a eliminar os
	movimentos desnecessários.
	
	\item \textit{Seiso}: \textbf{Senso de limpeza}. Designa a necessidade de
	manter o mais limpo possível o espaço de trabalho. A limpeza, nas empresas
	japonesas, é uma atividade diária. Ao fim de cada dia de trabalho, o
	ambiente é limpo e tudo é recolocado em seus lugares, tornando fácil saber
	o que vai aonde, e saber onde está aquilo o que é essencial. O foco deste
	procedimento é lembrar que a limpeza deve ser parte do trabalho diário, e
	não uma mera atividade ocasional quando os objetos estão muito desordenados.
	
	\item \textit{Seiketsu}: \textbf{Senso de Normalização}: Criar normas e
	sistemáticas em que todos devem cumprir. Tudo deve ser devidamente
	documentado. A gestão visual é fundamental para fácil entendimento de cada
	norma.
	
	\item \textit{Shitsuke} ou \textit{Shuukan}: \textbf{Senso de autodisciplina ou hábito}:
	Refere-se à manutenção e revisão dos padrões. Uma vez que os 4 Ss anteriores
	tenham sido estabelecidos, transformam-se numa nova maneira de trabalhar,
	não permitindo um regresso às antigas práticas. Entretanto, quando surge uma
	nova melhoria, ou uma nova ferramenta de trabalho, ou a decisão de implantação
	de novas práticas, pode ser aconselhável a revisão dos quatro princípios
	anteriores.
\end{enumerate}






\section{Inteligência Emocional}

Profissionais motivados e emocionalmente equilibrados são mais produtivos e
eficientes. Por isso o uso da Inteligência Emocional é um fator decisivo para
a obtenção de uma boa equipe de trabalho.

Daniel Goleman mapeia a Inteligência Emocional em cinco áreas de habilidades:

\begin{enumerate}
	\item \textbf{Auto-Conhecimento Emocional}: reconhecer um sentimento
	enquanto ele ocorre é a chave da inteligência emocional. A falta de
	habilidade em reconhecer nossos verdadeiros sentimentos deixa-nos a mercê
	de nossas emoções. Pessoas com esta habilidade são melhores pilotos de suas
	vidas.

	\item \textbf{Controle Emocional}: habilidade de lidar com seus próprios
	sentimentos, adequando-os para a situação. Pessoas pobres nesta habilidade
	afundam constantemente em sentimentos de incerteza, enquanto aquelas com
	melhor controle emocional tendem a recuperar-se mais rapidamente dos reveses
	e contratempos da vida.

	\item \textbf{Auto-Motivação}: Dirigir emoções a serviço de um objetivo é
	essencial para manter-se caminhando sempre em busca, para a automotivação,
	para manter-se sempre no controle e para manter a mente criativa na busca de
	soluções. Auto-controle emocional, sabendo praticar gratificação prorrogada
	e controlando impulsos. Pessoas que tem esta habilidade tendem a ser mais
	produtivas e eficazes, qualquer que seja seu empreendimento.

	\item \textbf{Reconhecimento de emoções em outras pessoas}: Empatia, outra
	habilidade que constrói auto-conhecimento emocional. Esta habilidade permite
	às pessoas reconhecer necessidades e desejos de outros, permitindo-lhes
	relacionamentos mais eficazes.

	\item \textbf{Habilidade em relacionamentos inter-pessoais}: A arte do
	relacionamento é, em grande parte, a habilidade de gerenciar sentimentos em
	outros. Esta habilidade é a base de sustentação de popularidade, liderança
	e eficiência interpessoal. Pessoas com esta habilidade são mais eficazes em
	tudo que é baseado na interação entre pessoas.
\end{enumerate}


%====================================



\printindex



\end{document}
