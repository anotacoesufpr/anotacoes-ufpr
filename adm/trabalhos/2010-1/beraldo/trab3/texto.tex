\documentclass[12pt]{article}
%\documentclass{abnt}

\usepackage[utf8]{inputenc}
\usepackage[brazil]{babel}



\begin{document}

\setlength{\parskip}{2ex}

% cabeçalho
\begin{tabular}[l]{ | p{10cm} | l | }
  \hline
  \multicolumn{2}{|c|}{CI069 - Administração de Empresas de Informática - Turma 2010-1} \\
  \hline
  Nome: Roberto Beraldo Chaiben & 
  GRR 20084213 \\
  \hline
  Título: 3 - Estrutura, missão, valores e visão organizacionais e como eles se sustentam em momentos de crise em uma empresa &
  Data: 01/06/2010 \\
  \hline
\end{tabular}


\vspace{1cm}


\section{Introdução}

Missão, visão e valores são conceitos essenciais para o progresso saudável de uma empresa. Sem essas definições, a obtenção de suas metas e objetivos
torna-se mais complexa, com mais obstáculos, dificuldades, desentendimentos e discussões entre os funcionários da organização.

Os conceitos de \textit{visão} e \textit{missão} são frequentemente confundidos por donos e funcionários de empresa. É importante distingui-los claramente.
Por isso, vamos definir cada um desses conceitos.


\subsection{Missão}

A missão de uma empresa consiste na razão da existência da empresa, seu objetivo primordial, o que ela deseja fazer e para quem fazê-lo. Ou seja, a missão
da empresa vai além da obtenção de lucro, que, logicamente, é um objetivo de qualquer empreendimento. Missão é o objetivo central da empresa, o papel social
da empresa, perante a sociedade.

A missão da empresa deve responder às questões abaixo:

\begin{itemize}

	\item Por que a empresa existe?
	\item O que a empresa faz?
	\item Para quem a empresa presta seus serviços?

\end{itemize}

Para exemplificar, vejamos a missão da empresa Google, encontrada em seu site oficial no Brasil:


\begin{quote}
\textit{Organizar as informações do mundo todo e torná-las acessíveis e úteis em caráter universal.}
\end{quote}


\subsection{Visão}

A visão de uma empresa consiste no maior objetivo do empreendimento, aonde ela deseja chegar a longo prazo. O enunciado da visão deve possuir a \textit{aspiração} e a 
\textit{inspiração} da empresa, ou seja, deve apresentar o objetivo principal e o que motiva a empresa a atingi-lo.

O enunciado da visão da empresa deve responder às questões abaixo:

\begin{itemize}

	\item No que a empresa deseja se tornar?
	\item Que rumos tomar para chegar ao objetivo?

\end{itemize}


\subsection{Valores}

Valores são o conjunto de de regras e valores éticos que norteiam a empresa em seu rumo a fim de executar sua Missão, na direção da Visão.

Os valores definem a forma como empreendedores e funcionários se relacionam entre si e como devem se portar perante seus clientes.

Os valores da empresa devem responder às seguintes questões:

\begin{itemize}

	\item Como os empregados devem se portar, individualmente?
	\item Como os empregados se relacionam entre si?
	\item Como os empregados se relacionam com os clientes?
	\item Como a empresa trata seus clientes?
	\item Qual a nossa responsabilidade frente à sociedade?
	\item Que valores, crenças ou princípios são importantes para a empresa fazer o que faz, para quem faz, e para o que ela quer se tornar?

\end{itemize}



\section{Momentos de Crise}

Momentos de crise econômica, como a ocorrida em 2008 e 2009, fazem com que muitas empresas declarem falência. Porém, são em momentos
como esse que podemos ver quais empresas estavam, de fato, olhando para o futuro.

Empreendimentos que superam uma grande crise são exemplos de organizações estáveis e com atitudes consolidadas.

Por isso, momentos de crise devem ser encarados como possibilidades de crescimento, inovação e renovação. Para superar crises, é necessário:

\begin{itemize}

	\item \textbf{Ter claras a visão e a missão da empresa}: apesar das dificuldades proporcionadas pela crise, o empreendimento nunca deve
	perder seu foco, ou seja, nunca desviar de sua missão e de sua visão;
	
	\item \textbf{Não abandonar seus valores}: durante a crise, os valores da empresa não devem ser deixados de lado. Eles devem continuar
	sendo seguidos, a fim de cumprir sua Missão, na direção da Visão;
	
	\item \textbf{Comunicação}: a empresa deve sempre manter a comunicação entre seus empreendedores e funcionários. Não se devem omitir
	dos funcionários problemas empresariais. Não adianta iludir os empregados, alegando que a empresa está em ótima situação. Quando a
	realidade for percebida, pode ser tarde demais para tomar atitudes. Comunicação é essencial em ambientes empresariais;
	
	\item \textbf{Investimento na equipe}: é necessário manter uma equipe competente e engajada no trabalho. É preciso investir em pessoas
	qualificadas, contratar bons funcionários, fornecer treinamentos, a fim de manter a melhor equipe possível;
	
	\item \textbf{Estimular a criatividade}: é preciso estimular a criatividade da equipe, para que possam enxergar inovações e renovações
	não somente para tempos de crise, mas para qualquer momento da empresa.

\end{itemize}





\section{Referências}

\begin{itemize}

	\item http://www.administradores.com.br/informe-se/producao-academica/a-importancia-da-missao-e-visao-dentro-da-organizacao/549/
	
	\item http://www.administradores.com.br/informe-se/artigos/missao-e-visao/21589/
	
	\item http://www.merkatus.com.br/10\_boletim/77.htm
	
	\item http://www.mariopersona.com.br/entrevista\_jornal\_exclusivo.html
	
	\item http://www.google.com.br/intl/pt-BR/corporate/

\end{itemize}

































\end{document}
