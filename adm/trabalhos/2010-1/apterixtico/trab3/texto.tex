\documentclass[12pt,a4paper]{article}
%\documentclass{abnt}

\usepackage[utf8]{inputenc}
\usepackage[brazil]{babel}

\newenvironment{separador}{
\begin{itemize}
  \setlength{\itemsep}{1pt}
  \setlength{\parskip}{0pt}
  \setlength{\parsep}{15pt}
}{\end{itemize}}

\begin{document}

\setlength{\parskip}{2ex}

% cabeçalho
\begin{tabular}[l]{ | p{10cm} | l | }
  \hline
  \multicolumn{2}{|c|}{CI069 - Administração de Empresas de Informática - Turma 2010-1} \\
  \hline
  Nome: Francisco Panis Kaseker & 
  GRR 20071909 \\
  \hline
  Título: 3 - Estrutura, missão, valores e visão organizacionais e como eles se sustentam em momentos de crise em uma empresa &
  Data: 01/06/2010 \\
  \hline
\end{tabular}


\vspace{1cm}

\section{Introdução}

Missão, visão e valores são conceitos essenciais para desenvolvimento saudável e sustentável de uma empresa. Sem essas definições, suas metas tornam-se mais complexas, causando mais desgastes à empresa e diminuindo a eficiência dela ao exercer suas atividades.

Os conceitos de \textit{visão} e \textit{missão} são frequentemente confundidos, mas dem ser distinguidos claramente. Cada seção abaixo explicará do que se trata cada um deles.

\subsection{Missão}

A missão de uma empresa consiste na sua razão de existir, o seu objetivo e qual é a sua razão de existência em relação à sociedade.

Para elucidar a idéia, vejamos a missão da empresa LG, encontrada em seu site oficial nacional:

\begin{quote}
\textit{Ser reconhecida como líder global no desenvolvimento de produtos e soluções inovadores, criando novas oportunidades de negócios.}
\end{quote}

\subsection{Visão}

A visão explica qual é o maior objetivo do empreendimento, basicamente o "sonho da organização". A Visão deve ser \textit{inspiradora}, \textit{clara} e \textit{concisa}, mas deve-se tomar cuidado, pois a Visão é mutável por natureza, diferentemente da Missão.

Para elucidar a idéia, vejamos a missão da empresa LG, encontrada em seu site oficial nacional:

\begin{quote}
\textit{Fornecer os melhores produtos e soluções para enriquecer a experiência humana na área de comunicação.}
\end{quote} 

\subsection{Valores}

Os Valores são os pontos que definem e orientam os princípios éticos de suas ações, garantindo eficiência nos resultados, ou seja, diminuindo a taxa de erros que toda empresa é acometida. Essa relação extende-se à relação entre empreendedores, seus funcionários e seus clientes. Os valores devem relacionar tópicos como: profissionalismo, ética, responsabilidade, reconhecimento, empreendedorismo e o espírito de equipe.

Para elucidar a idéia, vejamos a missão da empresa LG, encontrada em seu site oficial nacional:

\begin{quote}
\textit{Integridade. Nós acreditamos na integridade, na transparência e na verdade, sempre entregando com qualidade superior o que nos comprometemos a fazer. Dessa forma, conquistamos e mantemos a confiança e o respeito de nossos clientes, fornecedores, parceiros e colegas.}

\textit{Excelência.Nosso compromisso com a excelência está presente em todas as esferas de nosso trabalho e nosso relacionamento com consumidores, revendedores, distribuidores, parceiros, colegas e comunidades. O consumidor vem em primeiro lugar. Nós acreditamos em uma cultura de trabalho focada em nossos clientes; por isso priorizamos a busca por soluções aos seus problemas.}

\textit{Inovação. Com 70\% de nossos funcionários dedicados à área de Pesquisa e Desenvolvimento, nós conseguimos criar uma rica cultura de inovação. Esse ambiente é alimentado constantemente por nossa posição de destaque no mercado coreano de tecnologia e por nossa mentalidade empreendedora e inovadora que testa e incorpora as principais novidades em nossa rotina de trabalho.}

\textit{Liderança. Nossa estratégia é fornecer projetos que enriqueçam a vida das pessoas. Nossa paixão pela inovação é construída diariamente por nossos funcionários, gerando uma real vantagem competitiva para atingir a liderença de mercado.}

\textit{Respeito. Para nós, selecionar e desenvolver os melhores talentos é vital, uma vez que são as pessoas as responsáveis pelo sucesso de nossa companhia. Nós encorajamos a liberdade de comunicação entre os diferentes níveis hierárquicos da nossa empresa. Assim, diminuímos barreiras que impedem o surgimento de novas oportunidades de negócios e criamos um ambiente de trabalho cooperativo.}
\end{quote} 

\section{Momentos de Crise}

Nas maiores crises econômicas, mostram-se preparadas para contornar os problemas com solavancos de baixa intensidade aquelas empresas que se preocuparam em definir bem seus caminhos com sustentabilidade, com responsabilidade e com caixa suficiente tanto para suportar a crise quanto para corrigir erros da empresa perante aos seus clientes.

Deve-se ter bem claro quais são suas estratégios, seus planos de negócio e, embora óbvio mas muito importante, ter muito bem elucidado a pergunta "Qual é o seu negócio?" \textit{[Peter Drucker]}.

Não existe fórmula ou método mágico, mas ferramentas que tornam mais claras quais os caminhos e quais são as boas soluções para cada tipo de crise dentro da organização.

\begin{itemize}

	\item \textbf{Ter claras a Visão e a Missão da empresa}: Embora a Visão seja mutável (devido a sua natureza organizacional), a Missão não pode ser alterada e não pode-se perde-la de vista. Ela é um ítem fundamental para lembrar quem a organização é perante a sociedade;
	
	\item \textbf{Jamais abandonar os valores}: Os Valores da empresa são importantíssimas para mostrar qual os atributos que as ações da organização devem sempre estar acompanhadas, essa é a garantia de que a empresa não irá tomar um rumo de ações com seus funcionários, clientes, investimentos etc ineficiente ou até mesmo ilegal;
	
	\item \textbf{Investimento}: É muito importante manter sempre sua equipe e seu parque tecnológico bem treinado e/ou bem atualizado afim de torná-lo flexível às adversidades que toda organização está propensa a ter. Investimento em pessoas e idéias tornam a empresa mais dinâmica e inclusive mais bem vista. Não pode-se economizar em pessoas, são elas que corrigem ou fazem o que as máquinas não conseguem fazer;
	
	\item \textbf{Estimular a criatividade e o empreendedorismo}: Seus funcionários devem sempre ser estimulados a tomar decisões dentro de suas hierarquias e a torná-los empreendedores, sejam internos ou externos. Hoje a fronteira entre os dois tipos de empreendedores é muito tênua, afinal um empreendedor externo pode ser um empreendedor interno como fornecedor de serviços, equipamentos e idéias;

	\item \textbf{Concorrência}: Não deve-se temer a concorrência. Deve-se apenas enfrentá-la com um bom plano de negócios. A concorrência de empresas na verdade traz benefícios as ambas, pois dinamiza vários processos já que ambas irão querer ganhar mais recursos e ter cada vez menos custos. Sem concorrência a empresa evolui a passos pequenos. Deve-se enfrentá-la como desafios. E a idéia de concorrência deve ser passada também aos seus funcionários, principalmente os empreendedores internos, para melhorar a eficiência da organização. Mas deve-se ficar atento para garantir que sempre todos os valores da empresa sejam respeitados e que não exista pressão em cima dos funcionários, pois o stress é um dos grandes perigos da organização, pois esgota os seus melhores funcionários e todo o investimento feito neles acaba gerando retorno em um menor período de tempo;

	\item \textbf{Comunicação e relacionamento}: Jamais esconder dos seus clientes e dos seus funcionários os problemas que a organização enfrenta. A transparência deve ser um subitem essencial da responsabilidade da empresa. Isso pode ser um problema para as empresas com capital aberto, contudo torna claro as deficiências ou as adversidades das empresas e inibe a ação de especuladores que fazem a fuga de clientes e de capital aparecer;

\end{itemize}

\section{Referências}

\begin{separador}

	\item http://www.merkatus.com.br/10\_boletim/77.htm
	
	\item http://www.inovesempre.com.br/artigos.php?id=55
	
	\item http://www.aprenderafazer.com.br/a\_empresa\_missao.html
	
	\item http://www.lgnortel.com.br/open/sobre/visao
\end{separador}

\end{document}
