\documentclass[12pt]{article}
%\documentclass{abnt}

\usepackage[utf8]{inputenc}
\usepackage[brazil]{babel}



\begin{document}

\setlength{\parskip}{2ex}

% cabeçalho
\begin{tabular}[l]{ | p{10cm} | l | }
  \hline
  \multicolumn{2}{|c|}{CI069 - Administração de Empresas de Informática - Turma 2010-1} \\
  \hline
  Nome: Roberto Beraldo Chaiben & 
  GRR 20084213 \\
  \hline
  Título: 4 - Plano de Negócio &
  Data: 01/06/2010 \\
  \hline
\end{tabular}


\vspace{2cm}


\section{Introdução}

\subsection{Definição}

Um Plano de Negócios consiste num conjunto de planejamentos para criar um empreendimento e obter sucesso com ele.
Ele é um documento que descreve, por escrito, os objetivos de um negócio e quais passos devem ser dados para que
eles sejam alcançados, reduzindo riscos e incertezas. Um plano de negócio permite identificar e restringir seus
erros no papel, ao invés de cometê-los no mercado.

\subsection{Material de Referência}

Como referência para a criação deste trabalho, foi usada a apostila \textbf{Como elaborar um Plano de Negócios},
do SEBRAE, disponível na Internet, no link presente na seção \textbf{Referências} deste texto.

\subsection{Modelo Adotado para este Trabalho}

Para a criação do Plano de Negócio desta trabalho, considerei uma empresa de desenvolvimento de softwares.


\section{Plano de Negócio}

\subsection{Sumário Executivo}

\begin{itemize}

	\item \textbf{Resumo dos principais pontos do plano de negócio}: empresa de desenvolvimento de softwares
	em plataforma WEB para comércio eletrônico. Os principais clientes são empresas que desejam vender seus
	produtos e serviços via Internet;
	
	\item \textbf{Dados dos empreendedores, experiência profissional e atribuições}: Roberto Beraldo Chaiben,
	estudante de Ciência de Computação, com conhecimentos em desenvolvimento WEB há mais de 4 anos;
	
	\item \textbf{Dados do empreendimento}: razão social da empresa e seu CNPJ;
	
	\item \textbf{Missão da empresa}: Desenvolver soluções para comerciantes e promover a inclusão digital de
	micro e pequenas empresas;
	
	\item \textbf{Setores de atividades}: Prestação de Serviços;
	
	\item \textbf{Forma jurídica}: Empresário;

	\item \textbf{Enquadramento tributário}: Regime Normal Nacional;

	\item \textbf{Capital social}: consiste em todos os recursos, como dinheiro, equipamentos e ferramentas, utilizados para a
	criação do empreendimento;
	
	\item \textbf{Fonte de recursos}: é a fonte dos recursos que serão utilizados para a implantação da empresa. Inicialmente,
	os recursos podem ser próprios, de terceiros ou ambos. 

\end{itemize}


\subsection{Análise de Mercado}


\begin{itemize}

	\item \textbf{Estudo de Clientes}: Os clientes são empresas que desejam vender seus produtos e serviços via Internet. Essas empresas
	procurarão meus serviços pois desejam possuir espaço na Internet, a fim de expor e vender seus produtos e serviços, com o intuito de
	concorrer diretamente com seus concorrentes.

	\item \textbf{Estudo dos Concorrentes}: É preciso analisar a concorrência, pois são as empresas com as quais você vai competir diretamente.
	É necessário analisar suas instalações, seus produtos, formas de venda, pontos fortes e fracos. Para essa análise, considere os pontos abaixo:

	\begin{itemize}

		\item qualidade dos materiais empregados – cores, tamanhos, embalagem, variedade, etc.;
		\item preço cobrado;
		\item localização;
		\item condições de pagamento – prazos concedidos, descontos praticados, etc.;
		\item atendimento prestado;
		\item serviços disponibilizados – horário de funcionamento, entrega em domicílio, tele-atendimento, etc.;
		\item garantias oferecidas.

	\end{itemize}

	\item \textbf{Estudo dos Fornecedores}: é necessário possuir fornecedores de ambientes de execução de softwares
	WEB, ou seja, empresas que forneçam serviços de hospedagem de sites. É recomendado manter uma lista atualizada
	com todos eles, seus pontos fortes e fracos, características principais, para se adequar melhor a cada cliente da empresa.

\end{itemize}


\subsection{Análise de Marketing}

\begin{itemize}

	\item \textbf{Descrição dos Principais Produtos e Serviços}: Soluções em comércio eletrônico em plataforma WEB para empresas
	que desejam comercializar produtos e serviços via Internet;
	
	\item \textbf{Preço}: o preço do serviço deve ser calculado levando-se em consideração o custo de produção do serviço, ou seja,
	considerar os salários dos funcionários e a margem de lucro. Esse valor deve ser competitivo, a fim de não perder mercado para a
	concorrência;
	
	\item \textbf{Estratégias promocionais}: a principal forma de promover a empresa é fazer propaganda em meios de comunicação onde
	os clientes estão presentes em massa. A Internet é um dos melhores meios de propaganda, já que a maioria das empresas tem acesso
	fácil a ela. Propagandas em rádios são outra boa forma de divulgação: dispensa produções visuais e tem menor custo que propagandas
	em televisão. É importante possuir promoções e parcerias com os fornecedores, a fim de viabilizar menores preços, atraindo mais
	clientes.

	\item \textbf{Estrutura de Comercialização}: a prestação de serviço de criação de software não exige contato físico com o cliente.
	Por isso, é possível possuir representantes fora da empresa, fora da cidade sede, em todo Brasil ou mundo.
	
	\item \textbf{Localização do negócio}: escritório em local de fácil acesso aos funcionários e clientes;

\end{itemize}


\subsection{Plano Operacional}

\begin{itemize}

	\item \textbf{Layout}: Escritório com computadores e tecnologias recentes, para facilitar a prestação dos serviços. Mobília ergonômica,
	visando à saúde dos funcionários;
	
	\item \textbf{Capacidade produtiva/comercial/serviços}: estimativa da capacidade instalada da empresa, isto é, o quanto pode ser produzido
	ou quantos clientes podem ser atendidos com a estrutura existente. Com isso, é possível diminuir a ociosidade e a sobrecarga de tarefas;
	
	\item \textbf{Processos operacionais}: Após contato com o cliente e a contratação do serviço, analisam-se as necessidades do cliente, a fim
	de desenvolver a melhor solução a ele. Após isso os programadores desenvolvem o software de acordo com as especificações do cliente;
	
	\item \textbf{Necessidade de pessoal}: Inicialmente, uma equipe de dois programadores e um analista de sistemas podem suprir a demanda de serviços;

\end{itemize}




\subsection{Plano Financeiro}


\begin{itemize}

	\item \textbf{Estimativa dos investimentos fixos}: Compra de computadores e tecnologias recentes. Mobília adequada. Aluguel de escritório;
	
	\item \textbf{Capital de giro}: custo para o funcionamento normal da empresa. Isso envolve o pagamento dos funcionários, a margem de lucro
	da empresa, custo da prestação de serviço, custo de promoções e marketing;
	
	\item \textbf{Investimentos pré-operacionais}: Gastos realizados antes do início da empresa, como reforma do ambiente de trabalho, taxas
	de registro da empresa, divulgação;
	
	\item \textbf{Estimativa dos custos com mão-de-obra}: valor relacionado com o salário médio dos funcionários. Além disso, devem-se prever
	os gastos com impostos e benefícios, como INSS, FGTS, 13o. salário, férias etc;
	
\end{itemize}



\section{Referências}

\begin{itemize}

	\item http://www.biblioteca.sebrae.com.br/bds/bds.nsf/797332C6209B4B1283257368006\\FF4BA/\%24File/NT000361B2.pdf

\end{itemize}

































\end{document}
