\documentclass[12pt]{article}
%\documentclass{abnt}

\usepackage[utf8]{inputenc}
\usepackage[brazil]{babel}



\begin{document}

\setlength{\parskip}{2ex}

% cabeçalho
\begin{tabular}[l]{ | p{10cm} | l | }
  \hline
  \multicolumn{2}{|c|}{CI069 - Administração de Empresas de Informática - Turma 2010-1} \\
  \hline
  Nome: Francisco Panis Kaseker & 
  GRR 20071909 \\
  \hline
  Título: 4 - Plano de Negócio &
  Data: 01/06/2010 \\
  \hline
\end{tabular}


\vspace{2cm}


\section{Introdução}

\subsection{Definição}
%ok

Um Plano de Negócios é constituído por uma variedade de planejamentos afim de orientar a empresa a definir e atingir os seus objetivos. A principal vantagem do plano de negócios é aumentar a eficiência do projeto ao diminuir a taxa de erros cometidos e elucidar qual o caminho a ser seguido para atingir cada meta da organização. Um plano de negócios também pode considerar um caixa para enfrentar riscos e erros de produção ou em serviços, afim de evitar fundos negativos.

\subsection{Modelo Adotado para este Trabalho}
%ok
Uma empresa de revenda de hospedagem de sites.


\section{Plano de Negócio}

\subsection{Sumário Executivo}
%ok

\begin{itemize}

	\item \textbf{Resumo dos principais pontos do plano de negócio}: empresa de revenda de serviços de hospedagem estrangeiros no Brasil nos seguintes nichos: websites e comércio eletrônico.
	
	\item \textbf{Dados dos empreendedores, experiência profissional e atribuições}: Francisco Panis Kaseker,
	estudante de Bacharelado em Ciência de Computação, com cinco anos de experiência no gerenciamento de servidores e serviços web;
	
	\item \textbf{Dados do empreendimento}: razão social da empresa e seu CNPJ;
	
	\item \textbf{Missão da empresa}: Desenvolver soluções de baixo custo para o mercado nacional afim de maximizar a promoção da inclusão digital de
	micro e pequenas empresas no mercado internacional;
	
	\item \textbf{Setor de atividade}: Prestação de serviços (revenda de serviços estrangeiros);
	
	\item \textbf{Forma jurídica}: Empresário;

	\item \textbf{Enquadramento tributário}: Simples Nacional;

	\item \textbf{Capital social}: consiste em todos os recursos, como dinheiro, equipamentos e ferramentas, utilizados para a
	criação do empreendimento;
	
	\item \textbf{Fonte de recursos}: crédito em banco ou empréstimo entre pessoas físicas, já que o investimento inicial é baixo;

\end{itemize}


\subsection{Análise de Mercado}
%ok

\begin{itemize}

	\item \textbf{Estudo de Clientes}: Devido a grande relevância que a internet tem tido tanto no comércio quanto na propagação de informações e tecnologias e com a eficiência de sistemas de busca, as empresas tem procurado cada vez mais participar da rede mundial de computador para tentar explorar mercados entre outros estados e países.

	\item \textbf{Estudo dos Concorrentes}: É imprenscindível estudar os concorrentes especialmente para entender quais são os seus serviços de maior sucesso e compreender quais são os problemas que mais geram fuga de clientes.

	\begin{itemize}

		\item quantidade de recursos e tecnologias oferecidas;
		\item preço cobrado por cada plano de hospedagem;
		\item garantia de uptime dos serviços;
		\item serviços de backup dos dados dos clientes;
		\item tempo de atendimento do suporte técnico;
		\item nível de automatização das cobranças;

	\end{itemize}

	\item \textbf{Estudo dos Fornecedores}: Através de uma boa experiência deve-se escolher fornecedores estrangeiros desses serviços que possuam pesado investimento em infra-estrutura, redundância da rede elétrica e ethernet e com bom histórico no fornecimento dos seus serviços, além de escolher uma organização com um parque moderno.

\end{itemize}


\subsection{Análise de Marketing}
%ok
\begin{itemize}

	\item \textbf{Descrição dos Principais Produtos e Serviços}: Planos de hospedagem que garantem o fornecimento de recursos de acordo com cada necessidade dos websites, de acordo com o seu consumo de banda, espaço físico e níveis de \textit{Service Level Agreements}.
	
	\item \textbf{Preço}: Por causa da escolha de serviços estrangeiros é possível concorrer muito bem com os serviços nacionais; Os preços devem visar a concorrência de outros serviços que também revendam serviços estrangeiros, contudo deve também considerar a boa manutenção dos serviços e um suporte de exclência para os clientes, pois necessariamente eles não escolhem o serviço mais barato, mas o melhor atendimento pelo melhor preço em uma relação direta.
	
	\item \textbf{Estratégias promocionais}: Os recursos para propaganda e estratégias promocionais serão focados em sistemas de busca, portais de grande quantidade de acesso e parceria com websites bem acessados fornecendo serviçcos gratuitos em troca de propagandas. Utilização de recursos de media como mini-seriados ou desenhos em portais de conteúdo multimedia como Youtube e em redes sociais como Facebook e Twitter.

	\item \textbf{Estrutura de Comercialização}: Exclusivamente via web, com todos os processos automatizados em softwares com acesso completo a banco e sistemas de cartão de crédito. O cliente só precisa do apoio da organização para resolver problemas de faturamento eventuais que podem ocorrer devida as diferentes naturezas de cada sistema relacionado.
	
	\item \textbf{Localização do negócio}: escritório fechado e em um local seguro. A organização tem como foco ter poucos funcionários, assim tornando o local de trabalho pouco relavante.

\end{itemize}


\subsection{Plano Operacional}
%ok
\begin{itemize}

	\item \textbf{Layout}: Equipamentos topo de linha e teclados, mesas e cadeiras ergonômicas. Acesso a internet via banda larga de grande banda e ambiente controlado para o bem estar dos funcionários;
	
	\item \textbf{Capacidade produtiva/comercial/serviços}: Como o sistema de hospedagem de sites é extremamente otimizado, pode-se ter uma enorme quantidade de clientes por cada funcionário de suporte. O Suporte existe para corrigir problemas eventuais dos sistemas que não foram previstos, além de implementar as novas soluções, trazendo uma gigantesca capacidade produtiva à organização;
	
	\item \textbf{Processos operacionais}: O cliente escolhe um plano no site que pode ser criado dinâmicamente, cria uma conta, efetua o pagamento e o sistema automaticamente libera a conta para o uso. O não pagamento das novas faturas conclui-se em suspensão da conta. Todo o processo será automático e sem intervenção humana;
	
	\item \textbf{Necessidade de pessoal}: Apenas um administrador de contas e outro para programação do sistema. Se possível, a mesma pessoa para ambos. Bancos já fornecem bons gerenciadores financeiros e vários sistemas podem ser adquiridos através de pagamentos mensais de suas licenças. Poucos novos recursos precisam ser inseridos nos processos operacionais e, portanto, a necessidade inicial pode ser satisfeita por um único funcionário;

\end{itemize}

\subsection{Plano Financeiro}
%a fazer

\begin{itemize}

	\item \textbf{Estimativa dos investimentos fixos}: Compra de computadores e link com a internet, aluguel do local e serviços de contabilidade;
	
	\item \textbf{Capital de giro}: Dinheiro suficiente para manter ao menos três meses de investimento em propaganda e o pagamento das previsões de contas comuns (internet, eletricidade e serviços alugados);
	
	\item \textbf{Investimentos pré-operacionais}: Aluguel de servidores e contratação de equipes de manutenção e instalação de software do servidor;
	
	\item \textbf{Estimativa dos custos com mão-de-obra}: Devem-se prever os gastos com impostos e benefícios, como INSS, FGTS, 13o. salário, férias etc, contudo deve-se ter muita atenção com os contratos das empresas terceirizadas para provimento de determinados serviços; Muitas empresas serão contratadas para fornecer soluções para serviços online e pouquíssimos funcionários serão necessários.
	
\end{itemize}

\section{Referências}
%ok
\begin{itemize}

	\item Como elaborar um Plano de negócio, SEBRAE, http://www.biblioteca.sebrae.com.br/bds/bds.nsf/797332C6209B4B1283257368006\\FF4BA/\%24File/NT000361B2.pdf

	\item http://www.efetividade.net/2007/10/10/modelo-de-plano-de-negocios-como-fazer-o-seu-com-efetividade/

	\item http://www.planodenegocios.com.br/listaDinamica.asp?tipo\_tabela=artigo

	\item http://pt.wikipedia.org/wiki/Plano\_de\_neg\%C3\%B3cios

	\item http://www.unitrabalho.org.br/imagens/arquivos/arquivos/economiasolidaria/14-12-05/UFC\_OrientacaoEES.doc


\end{itemize}

\end{document}
