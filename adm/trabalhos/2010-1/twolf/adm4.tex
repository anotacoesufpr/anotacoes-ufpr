\documentclass[10pt]{article}
%\documentclass{abnt}

\usepackage[utf8]{inputenc}
\usepackage[brazil]{babel}



\begin{document}

\setlength{\parskip}{2ex}

% cabeçalho
\begin{tabular}[l]{ | p{10cm} | l | } 
  \hline
  \multicolumn{2}{|c|}{CI069 - Administração de Empresas de Informática - Turma 2010-1} \\
  \hline 
  Nome: Carlos Augusto Ligmanowski Junior & 
  GRR 20081238 \\
  \hline 
  Título: 4 - Plano de negócio & 
  Data: 01/06/2010 \\
  \hline 
\end{tabular}

\vspace{1cm}

\section{Introdução}

\subsection{Definição de Plano de Negócios}
Plano de negócio é um documento com o objetivo de estruturar as principais idéias e opções que o empreendedor analisará para decidir quanto à viabilidade da empresa a ser criada.

\subsection{Resumo executivo}
Treinamento online na área de TI. Modo robusto e prático , com professores capacitados e com vontade para passar a motivação que eles possuem com relação ao que fazem em um ambiente bem humorado possibilitando a descontração.

\section{Plano de Negócios}

\subsection{O que é o negócio}
Empresa de treinamento na área de TI.

\subsection{Produtos e serviços}
Treinamento online em diversas áreas de TI : Redes , Programação , Desenvolvimento Web , Segurança , etc.

\subsection{Razões que levam a crer que atingirá sucesso}
Treinamento robusto , com material e professores aptos a conseguir que o aluno crie um vínculo maior com o que faz. Tudo isso online , ou seja , custo baixo e prático.

\subsection{Oportunidades que pretende explorar}
Tanto empresas novas como veteranas no mercado. Com promoções , tentar mostrar que os treinamentos valem a pena, contribuindo para o desenvolvimento da empresa cliente.

\section{Administração}

\subsection{Quem administrará o negócio.}
Eu , Carlos Augusto Ligmanowski , quando possuir experiência para tal e sentir que estou pronto para assumir essa importante responsabilidade. 

\subsection{Experiências dos profissionais}
Os profissionais terão certificados reconhecidos internacionalmente no treinamento que darão. Administrador terá experiência mínima para ter competência para "tocar" o negócio com sucesso.

\subsection{Quantos funcionários terá e qual renumeração}
Terá um número máximo de 10 funcionários no começo. Ganharão uma quantia satisfatória e todos se empenharão para o desenvolvimento da empresa no começo, então salário inicial será o mínimo possível/satisfatório.

\subsection{Onde os funcionários trabalharão}
Em suas casas ou também no escritório da empresa, que não precisará ser grande, caso morem "perto" da instalação física.
 
\subsection{Estrutura Organizacional}
1 Administrador , 1 Chefe de Equipe , e 8 Funcionários. Todos fundamentais e com mesmo reconhecimento na empresa. Serviços extras serão terceirizados.

\subsection{Missão da Empresa}
Formar funcionários com paixão pelo que fazem de maneira robusta e prática.

\subsection{Visão da Empresa}
Em pouco tempo , possuir certificados reconhecidos internacionalmente e em consequência aprovação no setor.

\subsection{Valores}
Humildade e coragem para nunca desistir.

\subsection{Funcionários}
Mantê-los motivados em um ambiane de bom humor e tranquilidade , fazendo premiações sadias tanto para funcionários quanto para clientes.

\section{Sobre o mercado}

\subsection{Quem são os clientes em potencial}
Empresas em crescimento , empresas veterenas que precisam de atualização do pessoal e novas empresas.

\subsection{Divulgação}
Meio principal de divulgação: Internet , através de fóruns , site da empresa(com visual atrativo) e redes sociais(networking) e também trocas de serviços.

\subsection{Por que as empresas escolherão essa empresa??}
Porque oferecerá serviços baratos e robustos , com rapidez e praticidade.

\subsection{Parceiros} 
Comunidade do projeto GNU e empresas que oferecem serviços em troca dos nossos.

\subsection{Concorrentes}
Há um enorme número de concorrentes. Razão de sucesso deles: tempo de mercado , capacidade , etc.

\subsection{Diferencial}
Serviços on-line com disposição de material também on-line , com professores humorados e que fazem os alunos gostarem realmente do que estão fazendo.

\subsection{Delimitação geográfica}
Não há , pois os funcionários poderão trabalhar de suas casas caso desejem. Terá um escritório para representar a unidade física da empresa , onde funcionários locais poderão trabalhar , caso desejem.

\section{Sobre economia e finanças}

\subsection{Fontes do capital} 
De inicio , próprias, apostando na experiência e capacidade dos profissionais. Depois , será o fruto do trabalho.

\subsection{Projeção de faturamento , despesas e investimentos: 2 primeiros anos}
Previsão de faturamento de pelo menos 300\% do capital inicial, considerando as despesas. Despesas serão com os serviços tercerizados, manutenção do espaço físico , salários e manutenção dos serviços. Investimentos serão nos funcionarios , para oferecer serviços cada vez melhores.

\subsection{Tempo esperado para alcançar equilíbrio no negócio}
Tempo de 3 meses. Máximo de 6 meses , caso aconteça algum problema não previsto.

\subsection{Condições para começar a vender serviços}
Mínimo de 5 professores prontos para darem seu melhor.

\subsection{Valor do capital imobilizado em instalações e equipamentos}
Aluguel total que não ultrapasse 30\% do capital inicial para inciar a empresa e mantê-la por 6 meses. Equipamentos iniciais serão próprios dos funcionários.

\section{Mapas}

\subsection{Estado Atual}
Estágio de planejamento.

\subsection{Dificuldades esperadas}
Encontrar funcionários com a coragem de nunca desistir e com vontade de realizar.

\subsection{Soluções} 
Apresentar aos funcionários o lado bom e divertido , bem como o sentido verdadeiro na arte de passar o conhecimanto. Isso é dificil , mas como administrador tentarei me esforçar mais que todo mundo para poder passar esse ambiente próspero. 

\section{Conclusão}

A dificuldade principal é conseqüência da natureza dos planos de negócio: eles tratam de idéias ainda não realizadas, e assim não podem (de modo geral) se basear em históricos ou em estatśiticas próprias para realizar sua previsão. O grande desafio é conseguir obter dados de organizações semelhantes, ou extrapolar a partir de outros dados, de maneira consistente, objetiva e, principalmente, convincente. Mas o melhor de tudo é ver a criação se concertizar e atingir o objetivo. Só essa esperança talvez faça valer a pena o esforço para tudo acontecer. Fazer um plano desse precisa um mínimo de experiência , e eu como não a tenho , pude sentir a difculdade e também a quantidade de variáveis que fazem a empresa dar certo. É realmente muito esforço e dedicação!!

\section{Referências}

\begin{itemize}
	\item http://www.efetividade.net/2007/10/10/modelo-de-plano-de-negocios-como-fazer-o-seu-com-efetividade/
\end{itemize}


\end{document}
