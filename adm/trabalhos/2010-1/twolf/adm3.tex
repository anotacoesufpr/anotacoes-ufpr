\documentclass[11pt]{article}
%\documentclass{abnt}

\usepackage[utf8]{inputenc}
\usepackage[brazil]{babel}



\begin{document}

\setlength{\parskip}{2ex}

% cabeçalho
\begin{tabular}[l]{ | p{10cm} | l | } 
  \hline
  \multicolumn{2}{|c|}{CI069 - Administração de Empresas de Informática - Turma 2010-1} \\
  \hline 
  Nome: Carlos Augusto Ligmanowski Junior & 
  GRR 20081238 \\
  \hline 
  Título: 3 - Estrutura, missão, valores e visão organizacionais e como eles se sustentam em momentos de crise em uma empresa & 
  Data: 01/06/2010 \\
  \hline 
\end{tabular}

\vspace{1cm}

\section{Introdução}

Estrutura , missão , visão e valores são fundamentais na organização da empresa. São esses pilares que vão determinar como será o futuro da empresa e como que ela reagirá às adversidades gerais , como épocas de crise. 

\subsection{Missão de uma empresa}

É a identidade da empresa ou seja , aquilo que da direção e razão à existência da empresa.

Exemplos:

\textit{Missão FSF(Free Software Foundation) }: Preservar, proteger e promover a liberdade de usar, estudar, copiar, modificar e redistribuir programas de computador, e defende os direitos dos usuários de Software Livre. 

\textit{Missão da Samsung}: Tudo o que fazemos na Samsung é orientado por nossa missão: ser a melhor "empresa de soluções digitais". 

\subsection{Visão}

É o sonho da empresa. É aquilo que se espera ser em um determinado tempo e espaço. Jamais confundir Missão e Visão: a Missão é algo perene, sustentável enquanto a Visão é mutável por natureza, algo concreto a ser alcançado. A Visão deve ser inspiradora, clara e concisa, de modo que todos a sintam e queiram alcança-la.

Exemplo:

\textit{Visão da Samsung}: A Samsung é guiada por uma visão singular: liderar a revolução da convergência digital. 

\subsection{Valores}

Representam os princípios éticos que norteiam todas as suas ações. Geralmente, os valores compõem-se de regras morais que simbolizam os atos de seus fundadores, administradores e colaboradores em geral.

Exemplo: 

Valores na Samsung:

\begin{itemize}

	\item Pessoas:É muito simples: as pessoas fazem a empresa. Na Samsung, empenhamos todas as nossas forças para proporcionar aos nossos funcionários um leque de oportunidades que lhes permitam explorar todo o seu potencial.
	\item Excelência:Tudo o que fazemos na Samsung é orientado por uma inquebrantável paixão pela excelência, e por um compromisso firme de desenvolver os melhores produtos e serviços do mercado.
	\item Mudança:Na vertiginosa economia global de hoje, a mudança é constante e a inovação é essencial para a sobrevivência de uma empresa. Há 70 anos direcionamos nossos esforços para o futuro, prevendo as demandas e as necessidades do mercado, de forma que possamos conduzir nossa empresa para o sucesso a longo prazo.
	\item Integridade:Operar de maneira ética é a base do nosso negócio. Tudo o que fazemos é orientado por um guia moral que garante integridade, respeito por todas as partes interessadas e total transparência.
	\item Co-prosperidade:Uma empresa não terá sucesso a menos que crie prosperidade e oportunidade para terceiros. A Samsung se empenha em ser um cidadão corporativo social e ambientalmente responsável em todas as comunidades nas quais opera no mundo todo.
\end{itemize}	

\subsection{Estrutura}
Forma pela qual as atividades de uma empresa ou organização são divididas, organizadas e coordenadas. 

\section{Momentos de crise em empresas e sua sustentação}
Momentos de crise sempre existiram e sempre vão existir. Eles permitem a “limpa” do mercado, pois é na crise que desaparecem empresas que não conseguem se adaptar, as que conseguem permanecem e outras surgem com inovações que vêm criar novos paradigmas de mercado. Às vezes a crise surge em função de exigências externas. Há vários casos em que essas exigências obrigam a empresa a um reposicionamento, como por exemplo , uma simples troca de mão em uma rua movimentada. Crises de modo geral são imprevisíveis e estar preparado para elas é fundamental. 
Exemplo de uma crise:
 
\textbf{Crise Yahoo} 

\textit{Motivo :} Google.

\textit{Efeitos:} O diretor de tecnologia e mentor dos produtos do Yahoo!, Ari Balogh, deixa a companhia.  Bartz entra como executiva-chefe da companhia, em janeiro de 2009. Um mês depois houve uma profunda reorganização da estrutura de gestão do Yahoo! O Yahoo! fez um acordo de dez anos com a Microsoft para reduzir os custos e permitir revisão do foco da companhia.


\subsection{Principal medida a ser tomada para a empresa não entrar em crise junto ao mercado}

Nunca ter todos os ovos em uma cesta só. Qualquer empresa precisa de uma estratégia de desenvolvimento de novos produtos e de exploração de novos mercados. Estar preparado para uma coisa imprevisível é sempre muito sábio e maduro por parte da empresa. O mercado é implacável demais para permitir que algo sobreviva sem uma função clara e competitiva.

\subsection{Fatores para alcançar resultados positivos durante uma crise}

\textit{Estrutura , missão , visão e valores} ; esses são os pilares e fatores para enfrentar positivamente uma crise. Mario Persona, palestrante, escritor e estrategista ; cita em entrevista que habilidade de surfista é outro fator chave(ótima analogia). O surfista não espera a onda chegar para deitar na prancha e começar a remar. Ele começa a remar quando vê a onda no horizonte, procurando sincronizar ao máximo a velocidade da prancha com a onda que vem chegando. Se não fizer assim ele perde a onda. Do mesmo modo, não tem como ver o desvio ou a saída da estrada se dirigir olhando para o painel do carro. É preciso olhar as placas, os sinais do novo rumo e ir se preparando, procurando no horizonte um indício de que está chegando a hora de mudar. Marcio Persona diz: \textit{“Eu diria que o momento da crise é a prova que revela quem estava de olho no futuro e quem dirigia olhando para o painel da própria empresa, no carro da frente ou, o que é pior, para o retrovisor, preocupado com o sucesso ou fracasso — principalmente o fracasso — de sua concorrência"}.	

\subsection{Em tempos de crise , gestor de T.I deve investir em tecnologia ou cortar custos?}

De acordo com opiniões lidas em fóruns , sempre que a área de TI se focar em redução de custos e aumento de produtividade com uma visão de negócios, os investimentos jamais deveriam ser reduzidos e sim viabilizados dentro de um cenário de realidade e percepção da oportunidade. 
Alguns exemplos com soluções abaixo:

\begin{itemize}
	\item Intranet estática. Altos volumes de solicitação para mudanças de conteúdos. Solução com SharePoint Services (gratuíto).
	\item Alto custo com fitas de backup. Solução com ciclos mais eficiêntes de reutilização da mídia.
	\item Má utilização do servidor de arquivos. Solução do File Screener do Windows Server para bloquear tipos de documentos não desejados.
	\item Alto custo na recuperação de um backup. Solução com o uso de ShadowCopy para recuperação instantânea da informação.
	\item Dezenas de solicitações de relatórios. Solução com o Reporting Services (nativo do SQL Server) para desenvolvimento rápido dos relatórios.
	\item Processos não estruturados da empresa. Solução com uso de workflow do SharePoint (gratuíto) para automação de processos.
	\item Servidor de banco de dados lento. Remodelagem da estrutura e configuração, sem necessidade de aquisição de novo hardware.
\end{itemize}

\subsection{Medidas objetivas para superação da crise}	

Talvez não exista um remédio rápido para a crise. O que existe, é o que já foi citado: uma série de boas ações que deveriam existir em todas as empresas com ou sem crise.
Algumas coisa que requerem atenção são:
\begin{itemize}
	\item Desenvolva um plano estratégico de comunicação e marketing para a empresa e faça revisões periódicas;
	\item Explore ao máximo as competências da empresa e as habilidades de sua equipe;
	\item Invista nas pessoas, principalmente na contratação de novos talentos, reciclagem dos existentes e exclusão daqueles que de difícil adaptação a mudanças, é cruel mas necessário;
	\item Crie uma cultura organizacional visando levar a equipe a enxergar a empresa não como um emprego, mas como parte integrante de sua vida, ou seja, não como uma obrigação;
	\item Não economize em tecnologia e treinamento;
	\item Facilite uma cultura de empowerment para que cada um se sinta dono do seu pedaço ou das tarefas que lhe são designadas, pois dar valor às pessoas é obrigatório;
	\item Abra um canal de comunicação de baixo para cima(funcionários aos executivos principais) e esteja pronto a escutar o que os colaboradores pensam da empresa, do produto e dos serviços;
	\item Estimule a criatividade entre sua equipe, criando desafios de melhoria contínua e recompensando as melhores idéias.
	\item Dissemine a cultura de organização aprendiz para que cada membro da equipe seja capaz de enxergar oportunidades de mercado.
\end{itemize}

\section{Conclusão}

Nunca ter todos os ovos em uma cesta só e também não espere a bola no pé para fazer gol. Parece brincadeira mas é uma analogia muito séria. Estar pronto para desviar de pedras no caminho é um grande passo para o sucesso. 

\section{Referências}
\begin{itemize}
	\item http://www.mariopersona.com.br/entrevista\_jornal\_exclusivo.html 
	\item http://www.samsung.com/br/aboutsamsung/corporateprofile/\\visionmission.html 
	\item http://social.technet.microsoft.com/Forums/pt-BR/gerentetipt/\\thread/e54651ce-c7ad-49b8-8afd-d6522b3fa61a 
\end{itemize}

\end{document}
