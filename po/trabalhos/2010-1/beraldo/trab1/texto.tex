\documentclass{article}
\usepackage[utf8]{inputenc}
\usepackage[brazil]{babel}
\usepackage{hyperref}


\title{Trabalho 1 de Pesquisa Operacional - Situação 7}
\author{Roberto Beraldo Chaiben - GRR 20084213 \\ Carlos Augusto Ligmanowski Júnior - GRR 20081238}
\date{17/05/2010}


\begin{document}


\maketitle


\section{Tabela de Transporte}

\begin{tabular}{ | c | c | c | c | c | c | }


\hline
 & 1o. quadrimestre & 2o. q. & 3o. q. & Artificial & Capacidade \\
\hline
1o. quadrimestre & & & & & \\
normal           & 4    & 20   & BigM & 0 & 200 \\
extra            & BigM & BigM & BigM & 0 & 0 \\
terceirizado     & 10   & BigM & BigM & 0 & 200 \\
estocagem        & 2    & BigM & BigM & 0 & 250 \\
\hline
2o. quadrimestre & & & & & \\
normal       & BigM & 4    & 20   & 0 & 200 \\
extra        & BigM & BigM & BigM & 0 & 0 \\
terceirizado & BigM & 10   & BigM & 0 & 50 \\
estocagem    & BigM & 2    & BigM & 0 & 200 \\
\hline
3o. quadrimestre & & & & & \\
normal       & BigM & BigM & 4  & 0 & 200 \\
extra        & BigM & BigM & 6  & 0 & 40 \\
terceirizado & BigM & BigM & 10 & 0 & 50 \\
estocagem    & BigM & BigM & 2  & 0 & 200 \\
\hline
Demanda & 200-50 = 150 * & 300 & 500 & 640 & \\
\hline

\end{tabular} \\

\vspace{1cm}

*  50 = estoque inicial \\


total\_capacidade = 940 + x + y +z \\
total\_demanda = 950 \\


Como demanda\_total = capacidade\_total para se fazer o Algoritmo de Transporte: \\
Demanda artificial = 1590 - 950 = 640


\section{Modificações}

Usando os dados fornecidos pelo enunciado do problema, não há solução factível, 
uma vez que, mesmo a empresa operando no máximo de sua capacidade, ela não consegue
atender à demanda total.

A fim de corrigir esse problema, foram feitas as seguintes modificações:

\begin{itemize}

	\item Desconsideração do estoque final de 200 unidades;
	
	\item Aumento da capacidade de subcontratação dos quadrimestres 2 e 3 de 50 para 75.

\end{itemize}




\section{Modelagem}




\subsection{Variáveis de decisão}

$n_1$ = produção normal no 1o. quadrimestre \\
$n_2$ = produção normal no 2o. Quadrimestre \\
$n_3$ = produção normal no 3o. Quadrimestre \\
$e_3$ = produção extra no 3o. Quadrimestre \\
$s_1$ = produção terceirizada (subcontratação) no 1o. Quadrimestre \\
$s_2$ = produção terceirizada (subcontratação) no 2o. Quadrimestre \\
$s_3$ = produção terceirizada (subcontratação) no 3o. Quadrimestre \\
$z_1$ = produção para estocar no 1o. Quadrimestre \\
$z_2$ = produção para estocar no 2o. Quadrimestre \\
$z_3$ = produção para estocar no 3o. Quadrimestre \\
$a_1$ = quantidade atraso do 1o Quadrimestre no 2o Quadrimestre \\
$a_2$ = quantidade atraso do 2o Quadrimestre no 3o Quadrimestre \\


\subsection{Função objetivo}
MIN z = $4n_1$ + $4n_2$ + $4n_3$ + $6e_3$ + $10s_1$ + $10s_2$ + $10s_3$ + $2z_1$ + $2z_2$ + $2z_3$ + $20a_1$ + $20a_2$;



\subsection{Restrições}

\# Capacidade de produção normal \\
$n_1$ $\le$ 200; \\
$n_2$ $\le$ 200; \\
$n_3$ $\le$ 200; 

\vspace{1cm}

\# Capacidade de produção extra \\
$e_3$ $\le$ 40;

\vspace{1cm}


\# Subcontratação por quadrimestre \\
$s_1$ $\le$ 200; \\
$s_2$ $\le$ 75; \\
$s_3$ $\le$ 75; 

\vspace{1cm}

\# Restrições de estoque \\
$z_1$ $\le$ 250; \\
$z_2$ $\le$ 200; \\
$z_3$ $\le$ 200;

\vspace{1cm}

\# Demanda \\
$n_1$ + $s_1$ $=$ 150 + $z_1$ - $a_1$; \textit{\textbf{\# 200 - 50 (50 = estoque inicial)}} \\
$n_2$ + $s_2$ + $z_1$ $=$ 300 + $z_2$ + $a_1$ - $a_2$; \\
$n_3$ + $e_3$ + $s_3$ + $z_2$ $=$ 500 + $a_2$ + $z_3$;

\vspace{1cm}

\# No máximo atrasa o que tem de demanda \\
$a_1$ $\le$ 150; \\
$a_2$ $\le$ 300;


\section{Soluções}

$n_1$ = 200 \\
$n_2$ = 200 \\
$n_3$ = 200 \\
$e_3$ = 40 \\
$s_1$ = 100 \\
$s_2$ = 75 \\
$s_3$ = 75 \\
$z_1$ = 210 \\
$z_2$ = 185 \\
$z_3$ = 200 \\
$a_1$ = 0 \\
$a_2$ = 0 \\


\vspace{1cm}

Para a geração da solução, foi usado o programa GLPK (GNU Linear Programming Kit)\footnote{http://www.gnu.org/software/glpk/}


\section{Tabela de Decisões para cada Quadrimestre}

\begin{tabular}{| p{3cm} | p{10cm} |}

\hline
1o. Quadrimestre & Produzir 200 unidades como produção normal e 100, como subcontratada  \\
\hline
2o. Quadrimestre & Produzir 200 unidades como produção normal e 75, como produção subcontratada \\
\hline
3o. Quadrimestre & Produzir 200 unidades como produção normal, 40, como produção extra e 75, como produção subcontratada \\
\hline

\end{tabular}



\section{Análise da solução}

A solução é factível, uma vez que, no primeiro quadrimestre, precisa-se de 200 unidades. Como já se possui 50
unidades em estoque, basta produzir 150, como produção normal, que possui o menor custo. Como ainda é possível 
produzir mais, produz-se e estoca-se para o quadrimestre seguinte.

No segundo quadrimestre, produz-se, além das 200 unidades da produção normal, 75 em produção subcontratada, gerando menor custo.

No terceiro quadrimestre, precisa-se de 500 unidades, pois o estoque final foi desconsiderado.  Produz-se
200 em produção normal, 40 em produção extra e 75 em subcontratada.






\end{document}
