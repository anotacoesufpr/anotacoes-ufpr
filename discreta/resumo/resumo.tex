\documentclass[a4paper,11pt]{article}
\usepackage[pdftex,pagebackref=true,colorlinks=true,linkcolor=blue,unicode]{hyperref}
\usepackage{amsmath}
\usepackage{amssymb}
\usepackage{amsthm}
\usepackage[brazil]{babel}

\author{Victor Tarabola Cortiano}
\title{CI237: Matem\'{a}tica Discreta}

\theoremstyle{definition} \newtheorem{definicao}{Def}
\theoremstyle{definition} \newtheorem{exemplo}{Exemplo}
\theoremstyle{plain}      \newtheorem{teorema}{Teorema}
\theoremstyle{remark}     \newtheorem*{corolario}{Corol\'{a}rio}


\begin{document}


\maketitle
\tableofcontents

\newpage



\section{Informa\c{c}\~{o}es sobre a disciplina}

\begin{enumerate}

\item Programa
\begin{itemize}
\item Fundamentos
\item Indu\c{c}\~{a}o
\item Recurs\~{a}o
\item Recorr\^{e}ncias
\item Analise Assint\'{o}tica
\item Contagem
\end{itemize}
\end{enumerate}

\section{Conceitos B\'{a}sicos}

\begin{definicao}
Chao: $\lfloor x \rfloor = max\{z \in \mathbb{Z}, z \leq x\}$
\end{definicao}
\begin{definicao}
Teto: $\lceil  x \rceil  = min\{z \in \mathbb{Z}, z \geq x\}$
\end{definicao}
\begin{definicao}
$\mathbb{A}^\mathbb{B} = \{f, f: \mathbb{B} \longrightarrow \mathbb{A}\}$
\end{definicao}

\begin{teorema}
$\lfloor x \rfloor$ \'{e} o \'{u}nico inteiro que satisfaz
$x - 1 < \lfloor x \rfloor \leq x$
\end{teorema}
\begin{teorema}
$\lceil x \rceil$ \'{e} o \'{u}nico inteiro que satisfaz
$x \leq \lceil x \rceil < x + 1$
\end{teorema}

\section{Recorr\^{e}ncias}

A maioria das fun\c{c}oes tratadas doravante ser\~{a}o do tipo
$f: \mathbb{N} \longrightarrow \mathbb{C}$.

\begin{exemplo}
Seja $l: \mathbb{N} \longrightarrow \mathbb{N}$ o n\'{u}mero de
digitos na representa\c{c}\~{a}o bin\'{a}ria de n
\end{exemplo}

\begin{teorema}
\begin{equation*}
l(n) =
\begin{cases}
0 & \text{se } n = 0, \\
l(\lfloor \frac{n}{2} \rfloor) + 1 & \text{se } n \neq 0
\end{cases}
\end{equation*}
\end{teorema}

\begin{corolario}
$\forall n > 0 \qquad l(n) = \lfloor \log n \rfloor + 1 $
\end{corolario}

\begin{exemplo}
Seja $b: \mathbb{N} \longrightarrow \mathbb{N}$ o n\'{u}mero de
d\'{i}gitos 1 na representa\c{c}\~{a}o bin\'{a}ria de n.
\end{exemplo}

\begin{teorema}
\begin{equation*}
b(n) =
\begin{cases}
0 & \text{se } n = 0, \\
b(\lfloor \frac{n}{2} \rfloor) + n \mod 2 & \text{se } n \neq 0
\end{cases}
\end{equation*}
\end{teorema}

\begin{definicao}
Seja $f: \mathbb{A} \longrightarrow \mathbb{A}$. Para todo
$n \in \mathbb{N}$, definimos $f^n: \mathbb{N} \longrightarrow \mathbb{N}$
como a fun\c{c}\~{a}o dada por:
\begin{equation*}
f^n(a) =
\begin{cases}
a & \text{se } n = 0, \\
f\Big(f^{n-1}(a)\Big) & \text{se } n > 0
\end{cases}
\end{equation*}
\end{definicao}

\begin{corolario}
Se $f(x) = \lfloor \frac{x}{k} \rfloor$ ent\~{a}o
$f^n(x) = \lfloor \frac{x}{k^n} \rfloor$, com $k \in \mathbb{N}$
\end{corolario}

\begin{definicao}
Uma recorrencia \'{e} uma fun\c{c}\~{a}o definida recursivamente.
\end{definicao}

\begin{exemplo}
Seja $f(n) = f(n-1) + 1 \forall n > 0$. Queremos uma express\~{a}o
n\~{a}o recursiva para $f$.

Se $n > 0$, ent\~{a}o

\begin{eqnarray*}
f(n) & = & f(n-1) + 1 = \\
& = & \Big(f(n-1-1) + 1\Big) = \\
& = & \cdots = \\
& = & f(n-u) + u
\end{eqnarray*}

$u$ \'{e} o menor inteiro tal que $n - u \leq 0$. Em outras palavras: \\
$u = min\{k \in \mathbb{N}, n -k \leq 0\} = n$. Portanto: \\
$f(n) = f(n-u) + u = f(0) + n$
\end{exemplo}

\begin{teorema}
Seja $f(n) = f\Big(h(n)\Big) + s(n)$ para todo $n > n_0$.
Ent\~{a}o: \\
\begin{equation*}
f(n) = f\Big(h^u(n)\Big) + \sum_{i=0}^{u-1} s\Big(h^i(n)\Big)
\end{equation*}
Na qual $u = min\{k \in \mathbb{N}, h^k(n) \leq n_0\}$
\end{teorema}

\begin{exemplo}
Uma progress\~{a}o aritm\'{e}tica \'{e} uma fun\c{c}\~{a}o que satisfaz
a recorr\^{e}ncia $f(n) = f(n-1) + r$. Usando a nota\c{c}\~{a}o anterior
temos: $h(n) = n - 1$ e $s(n) = r$ \\
Temos que $h^k(n) = n - k$.
\begin{eqnarray*}
u & = & min\{k \in \mathbb{N}, h^k(n) \leq n_0\} = \\
& = & min\{k \in \mathbb{N}, n - k \leq n_0\} = \\
& = & min\{k \in \mathbb{N}, n - n_0 \leq k\} = \\
& = & n - n_0
\end{eqnarray*}
Portanto,
\begin{eqnarray*}
f(n) & = & f\Big(h^{n-n_0}(n)\Big) + \sum_{i = 0}^{n - n_0 - 1}
s\Big(h^i(n)\Big) = \\
& = & f\Big(n - (n - n_0)\Big) + \sum_{i = 0}^{n - n_0 - 1} r = \\
& = & f(n_0) + (n-n_0)r
\end{eqnarray*}
\end{exemplo}

\begin{teorema}
Seja $f(n) = m(n)f\Big(h(n)\Big)$ para todo $n > n_0$.
Ent\~{a}o: \\
\begin{equation*}
f(n) = f\Big(h^u(n)\Big)\prod_{i=0}^{u-1}m\Big(h^i(n)\Big)
\end{equation*}
Na qual $u = min\{k \in \mathbb{N}, h^k(n) \leq n_0\}$
\end{teorema}

\begin{exemplo}
Uma progress\~{a}o geom\'{e}trica \'{e} uma fun\c{c}\~{a}o que satisfaz
a recorr\^{e}ncia $f(n) = qf(n-1)$.  Usando a nota\c{c}\~{a}o anterior
temos: $h(n) = n - 1$ e $m(n) = q$ \\
Temos que $h^k(n) = n - k$.
\begin{eqnarray*}
u & = & min\{k \in \mathbb{N}, h^k(n) \leq n_0\} = \\
& = & min\{k \in \mathbb{N}, n - k \leq n_0\} = \\
& = & min\{k \in \mathbb{N}, n - n_0 \leq k\} = \\
& = & n - n_0
\end{eqnarray*}
Portanto,
\begin{eqnarray*}
f(n) & = & f\Big(h^{n-n_0}(n)\Big)\prod_{i=0}^{u-1} m\Big(h^i(n)\Big) = \\
& = & f\Big(n - (n - n_0)\Big)\prod_{i = 0}^{n - n_0 - 1} q = \\
& = & f(n_0) q^{n - n_0}
\end{eqnarray*}
\end{exemplo}

\begin{teorema}
Seja $f(n) = m(n)f\Big(h(n)\Big) + s(n)$ para todo $n > n_0$.
Ent\~{a}o: \\
\begin{equation*}
f(n) = f\Big(h^u(n)\Big)\prod_{i=0}^{u-1}m\Big(h^i(n)\Big) +
\sum_{i=0}^{u-1}s\Big(h^i(n)\Big)\prod_{j=0}^{i-1}m\Big(h^j(n)\Big)
\end{equation*}
Na qual $u = min\{k \in \mathbb{N}, h^k(n) \leq n_0\}$
\end{teorema}

\begin{teorema}
Dado um polin\^{o}mio de ra\'{i}zes $r_1, \cdots, r_{l}$ e respectivas
multiplicidades $m_1, \cdots, m_{l}$, sua base \'{e} dada por:
\begin{equation*}
\Big\{r_1^n, nr_1^n,\cdots,n^{m_1-1}r_1^n,
r_2^n, nr_2^n,\cdots,n^{m_2-1}r_2^n, \cdots,
r_{l}^n, nr_{l}^n,\cdots,n^{m_{l}-1}r_{l}^n\Big\}
\end{equation*}
\end{teorema}

\begin{definicao}
Definimos como Recorr\^{e}cia Linear Homog\^{e}nia toda fun\c{c}\~{a}o
da forma:
\begin{equation*}
f(n) = \sum_{i=1}^{k} a_if(n-i) \qquad \forall n \geq k
\end{equation*}
\end{definicao}

\begin{exemplo}
A Sequ\^{e}ncia de Fibonacci \'{e} um caso espec\'{i}fico de
Recorr\^{e}cia Linear Homog\^{e}nia:
\begin{equation*}
f(n) =
\begin{cases}
f(n-1) + f(n-2) & \text{se } n > 2, \\
1 & \text{se } n = 1 \\
0 & \text{se } n = 0 \\
\end{cases}
\end{equation*}
\end{exemplo}

\begin{corolario}
A Sequ\^{e}ncia de Fibonacci satisfaz: \\
$f(n) + g(n) = (f+g)(n-1)+(f+g)(n-2)$ \\
$(zf)(n) = (zf)(n-1) + (zf)(n-2)$
\end{corolario}

\begin{definicao}
Seja $F = \{f \in \mathbb{C}^\mathbb{N}, f(n) = f(n-1) + f(n-2)$
\end{definicao}

\begin{corolario}
O conjunto $F$ \'{e} um subespa\c{c}o vetorial de
$(\mathbb{C}^\mathbb{N},+)$
\end{corolario}

Queremos encontrar uma base para F: \\
Seja $r \in \mathbb{C}$ e $f: \mathbb{N} \longrightarrow \mathbb{C}$
dada por $f(n) = r^n$. Assumindo que $f \in F$, ent\~{a}o
$f(n)=f(n-1)+f(n-2)$ para todo $n \geq 2$, ou seja:
$r^n = r^{n-1} + r^{n-2} \therefore r^n - r^{n-1} - r^{n-2} = 0$
$\therefore r^{n-2}(r^2 -r -1) = 0 \therefore r^{n-2} = 0$ ou
$r^2 -r -1 = 0$

Resolvendo a equa\c{c}\~{a}o do segundo grau, temos que:
\begin{equation*}
r \in \Big\{0, \frac{1 - \sqrt{5}}{2}, \frac{1 + \sqrt{5}}{2}\Big\}
\end{equation*}

Fazendo $r_1 = \frac{1 - \sqrt{5}}{2}$ e $r_2 = \frac{1 + \sqrt{5}}{2}$

Ent\~{a}o $f_1 \in F$ e $f_1 \in F$, na qual $f_1(n)=r_1^n$
e $f_2(n)=r_2^n$.

Como $f_1$ e $f_2$ s\~{a}o LI em $(\mathbb{C}^\mathbb{N},+)$,
ent\~{a}o $\{f_1,f_2\}$ \'{e} uma base de $F$.

Consequentemente, $\forall f \in F, \exists c_1,c_2 \in \mathbb{C}$
tais que $f(n) = c_1f_1(n) + c_2f_2(n)$

Em particular:

\begin{equation*}
\begin{cases}
f(0) = c_1f_1(0) + c_2f_2(0) \\
f(1) = c_1f_1(1) + c_2f_2(1)
\end{cases}
\end{equation*}
Ou seja:
\begin{equation*}
\begin{cases}
f(0) = c_1r_1^0 + c_2r_2^0 = c_1 + c_2 \\
f(1) = c_1r_1^1 + c_2r_2^1 = c_1r_1 + c_2r_2
\end{cases}
\end{equation*}

Em particular, na Sequ\^{e}ncia de Fibonacci temos $f(0) = 0$ e
$f(1) = 1$, portanto:
\begin{equation*}
\begin{cases}
c_1 + c_2 = 0\\
c_1r_1 + c_2r_2 = 1
\end{cases}
\end{equation*}

Portanto, $c_1 = \frac{-1}{r_2-r_1}$ e $c_2 = \frac{1}{r_2-r_1}$

Como $r_2 - r_1 = \frac{1+\sqrt{5}}{2} - \frac{1-\sqrt{5}}{2} = $
$\sqrt{5}$, Ent\~{a}o:\\
$c_1 = - \frac{1}{\sqrt{5}} = - \frac{\sqrt{5}}{5}$ e
$c_2 = \frac{1}{\sqrt{5}} = \frac{\sqrt{5}}{5}$. Portanto:

\begin{eqnarray*}
f(n) & = & c_1f(n-1) + c_2f(n-2) = \\
& = & - \frac{\sqrt{5}}{5}\Big(\frac{1-\sqrt{5}}{2}\Big)^n +
\frac{\sqrt{5}}{5}\Big(\frac{1+\sqrt{5}}{2}\Big)^n = \\
& = & \frac{\sqrt{5}}{5} \bigg[\Big(\frac{1+\sqrt{5}}{2}\Big)^n -
\Big(\frac{1-\sqrt{5}}{2}\Big)^n\bigg]
\end{eqnarray*}

\begin{definicao}
Se $r$ \'{e} raiz do polin\^{o}mio $P$, ent\~{a}o a multiciplade
de $r$ \'{e}:
\begin{equation*}
max\{k \in \mathbb{N}, (x-r)^k | P\}
\end{equation*}
\end{definicao}

\begin{corolario}
Sejam $r_1,r_2,\cdots,r_l$ as distintas raizes do polin\^{o}mio
caracter\'{i}stico de $R(a_1, \cdots, a_k)$, com multiplicidades
$m_1,m_2,\cdots,m_l$, respectivamente. O conjunto
$\{n^j r_i^n, 1 \leq i \leq l$ e $0 \leq j < m\}$ \'{e} uma base
de $R(a_1, \cdots, a_k)$.
\end{corolario}

\begin{corolario}
Se $f(n)$ satisfaz uma Recorr\^{e}cia Linear Homog\^{e}nia para todo
$n \geq k$, ent\~{a}o:
\begin{equation*}
f(n) = \sum_{i=1}^{l}\sum_{j=0}^{m_i-1} c_{ij} n^j r_i^n
\end{equation*}
Na qual $r_i$ s\~{a}o as ra\'{i}zes, $m_i$ as multiplicidades e
$c_{ij}$ s\~{a}o dados pela solu\c{c}\~{a}o de um sistema linear.
\end{corolario}

\begin{definicao}
Uma Recorr\^{e}cia Linear n\~{a}o Homog\^{e}nia \'{e} uma
recorr\^{e}ncia linear da forma:
\begin{equation*}
f(n) = \sum_{i=1}^{k} a_i f(n-i) + g(n) \qquad \forall n \geq k
\end{equation*}
\end{definicao}

\begin{teorema}
A fun\c{c}\~{a}o $f(n) = c n^k r^n$ com $k \in \mathbb{N}$ e
$c,r \in \mathbb{C}$ satisfaz uma RLH cujo polin\^{o}mio
caracter\'{i}stico \'{e} $(x-r)^{k+1}$.
\end{teorema}

\begin{teorema}
\begin{equation*}
\sum_{i=0}^{n} i 2^i = 2^n(n-1) + 2
\end{equation*}
\end{teorema}

\begin{proof}
\begin{equation*}
\text{Seja } s(n) = \sum_{i=0}^{n}i2^i =
\sum_{i=0}^{n-1}i2^i + n2^n = s(n-1) + n2^n
\end{equation*}
Portanto $s$ satisfaz uma RLnH. $n2^n$ satisfaz uma RLH cujo
polin\^{o}mio caracter\'{i}stico \'{e} $(x-2)^2$, ent\~{a}o
$s(n)$ satisfaz uma RLH cujo polin\~{o}mio caracter\'{i}stico
\'{e}: $(x-1)(x-2)^2$ e, portanto,
$s(n) = c_{10}1^n + c_{20}2^n + c_{21}n2^n$. Logo:
\begin{equation*}
\begin{cases}
s(0) = c_{10}1^0 + c_{20}2^0 + c_{21}(0)2^0 \\
s(1) = c_{10}1^1 + c_{20}2^1 + c_{21}(1)2^1 \\
s(2) = c_{10}1^2 + c_{20}2^2 + c_{21}(2)2^2
\end{cases}
\end{equation*}
Ou seja,
\begin{equation*}
\begin{cases}
0 = c_{10} + c_{20} \\
2 = c_{10} + 2c_{20} + 2c_{21} \\
10 = c_{10} + 4c_{20} + 8c_{21}
\end{cases}
\end{equation*}
Portanto, $c_{10} = 2, c_{20} = -2, c_{21} = 2$, Ent\~{a}o:\\
\begin{equation*}
s(n) = 2 -2(2^n) + 2n2^n = 2^n(n-1)+2=(n-1)2^n + 2
\end{equation*}
\end{proof}

\begin{corolario}
Se $f$ satisfaz uma RLH cujo polin\^{o}mio caracter\'{i}stico
\'{e} P, ent\~{a}o:
\begin{equation*}
s(n) = \sum_{i=0}^{n} f(i) \text{ satisfaz uma RLH cujo polin\^{o}mio
caracter\'{i}stico \'{e} } P(x-1)
\end{equation*}
\end{corolario}

\begin{proof}
\begin{equation*}
\text{Como } s(n) = \sum_{i=0}^{n}f(i) = \sum_{i=0}^{n-1}f(i) + f(n)=
s(n-1) + f(n)
\end{equation*}
Ent\~{a}o, $s$ satisfaz a RLnH $s(n)=s(n-1)+f(n)$. Portanto,
$s$ satisfaz uma RLH cujo polin\^{o}mio caracter\'{i}stico \'{e}
$P(x-1)$
\end{proof}

\begin{exemplo}
Soma de uma Progress\~{a}o Aritm\'{e}tica:
\begin{equation*}
s(n) = \sum_{i-0}^{n} f(i)
\end{equation*}
cujo polin\^{o}mio caracter\'{i}stico \'{e} $(x-1)^2(x-1)=(x-1)^3$.
Portanto: \\ $s(n) = c_{10}1^n + c_{11}n1^n + c_{12}n^21^n =$
$c_{10}+nc_{11}+n^2c_{12}$. Ent\~{a}o:
\begin{equation*}
\begin{cases}
f(0) = c_{10}+(0)c_{11}+(0)^2c_{12} \\
f(0) + f(1) = c_{10}+(1)c_{11}+(1)^2c_{12} \\
f(0)+f(1)+f(2) = c_{10}+(2)c_{11}+(2)^2c_{12}\\
\end{cases}
\end{equation*}
Em outras palavras:
\begin{equation*}
\begin{cases}
f(0) = c_{10} \\
f(0) + f(0) + r = c_{10}+c_{11}+c_{12} \\
f(0)+f(0)+r+f(0)+2r = c_{10}+2c_{11}+4c_{12}
\end{cases}
\end{equation*}
Ou seja:
\begin{equation*}
\begin{cases}
f(0) = c_{10} \\
2f(0) + r = c_{10}+c_{11}+c_{12} \\
3f(0)+3r = c_{10}+2c_{11}+4c_{12}
\end{cases}
\end{equation*}
Portanto, $c_{10}=f(0),c_{11}=f(0)+\frac{r}{2},c_{12}=\frac{r}{2}$.
Ent\~{a}o:
\begin{eqnarray*}
f(n) & = & f(0)+\Big(f(0) + \frac{r}{2}\Big)n + (\frac{r}{2})n^2 = \\
& = & f(0)+nf(0)+n\frac{r}{2}+n^2\frac{r}{2} = \\
& = & f(0)(n+1) + \frac{n}{2}(n-1)r
\end{eqnarray*}
\end{exemplo}

\begin{exemplo}
O n-\'{e}simo termo de uma Progress\~{a}o Geom\'{e}trica pode ser
descrito pela recorr\^{e}ncia $f(n)=rf(n-1)$, cujo polin\^{o}mio
caracter\'{i}stico \'{e} $(x-r)$. Base: $\{(r^n)\}$. Portanto:
$f(n) = c_{10}r^0 = c_{10}.$ \\
Logo, $f(n)=f(0)r^n$.
\end{exemplo}

\begin{exemplo}
Soma de uma Progress\~{a}o Geom\'{e}trica:
\begin{equation*}
s(n) = \sum_{i-0}^{n} f(i)
\end{equation*}
cujo polin\^{o}mio caracter\'{i}stico \'{e} $(x-r)(x-1).$
Base: $\{(r^n,1^n)\}$. Portanto:
\begin{equation*}
s(n) = c_{10}1^n + c_{20}r^n = c_{10} + c_{20}r^n
\end{equation*}
\begin{equation*}
\begin{cases}
f(0) = c_{10} + c_{20}r^0 \\
f(0) + f(1) = c_{10}+c_{20}r^1 \\
\end{cases}
\end{equation*}
Em outras palavras:
\begin{equation*}
\begin{cases}
f(0) = c_{10}+c_{20} \\
f(0) + f(0)r = c_{10}+c_{20}r \\
\end{cases}
\end{equation*}
Sou seja:
\begin{equation*}
\begin{cases}
f(0) = c_{10}+c_{20} \\
f(0)(r+1) = c_{10}+c_{20}r \\
\end{cases}
\end{equation*}
Portanto, $c_{10}=-\frac{f(0)}{r-1}$ e $c_{20}=-\frac{f(0)r}{r-1}$.
Ent\~{a}o:
\begin{equation*}
s(n) = -\frac{f(0)}{r-1} + \frac{f(0)r}{r-1}r^n =
f(0)\Big(\frac{r^{n+1}-1}{r-1}\Big)
\end{equation*}
\end{exemplo}

\section{\'{A}rvores Bin\'{a}rias}

\begin{definicao}
\'{A}rvores Bin\'{a}ria \'{e} uma \'{a}rvore (representada por $\lambda$)
ou um par $\Big(E(t),D(t)\Big)$, no qual $E(t)$ e $D(t)$ s\~{a}o
\'{a}rvores bin\'{a}rias. Definimos $\mathbb{B}$ como o conjunto de todas
as \'{a}rvores bin\'{a}rias.
\end{definicao}

\begin{definicao}
O tamanho de uma \'{a}rvore bin\'{a}ria $t$ \'{e} dada por:
\begin{equation*}
|t| =
\begin{cases}
0 & \text{se } t = \lambda, \\
|E(t)| + |D(t)| + 1 & \text{se } t \neq \lambda
\end{cases}
\end{equation*}
\end{definicao}

\begin{definicao}
A altura de uma \'{a}rvore bin\'{a}ria $t$ \'{e} dada por:
\begin{equation*}
h(t) =
\begin{cases}
0 & \text{se } t = \lambda, \\
1 + max\bigg\{h\Big(E(t)\Big),h\Big(D(t)\Big)\bigg\}
& \text{se } t \neq \lambda
\end{cases}
\end{equation*}
\end{definicao}

\begin{definicao}
\begin{equation*}
h^{-}(n) = min\{h(t), |t| = n\}
\end{equation*}
\begin{equation*}
h^{+}(n) = max\{h(t), |t| = n\}
\end{equation*}
\end{definicao}

\begin{corolario}
\begin{equation*}
h^{-}(|t|) \leq h(t) \leq h^{+}(|t|) \qquad
\forall t \in \mathbb{B}
\end{equation*}
\end{corolario}

\begin{teorema}
\begin{equation*}
h^{+}(n) = n
\end{equation*}
\end{teorema}

\begin{proof}
\begin{eqnarray*}
h^{+}(n) & = & max\{h(t), |t| = n\} = \\
& = & max\Bigg\{1 + max\bigg\{h\Big(E(t)\Big),
h\Big(D(t)\Big)\bigg\}, |t| = n\Bigg\} = \\
& = & 1 + max\{h(t), |t| < n\} = \\
& = & 1 + max\{h^{+}(k), k < n\} = \qquad
\text{ como } h^{+} \text{ \'{e} crescente} \\
& = & 1 + h^{+}(n-1) = \\
& = & n
\end{eqnarray*}
\end{proof}

\begin{corolario}
\begin{equation*}
h^{-}(|t|) \leq h(t) \leq |t| \qquad
\forall t \in \mathbb{B}
\end{equation*}
\end{corolario}

\begin{definicao}
\begin{equation*}
T^{-}(n) = min\{|t|, h(t) = n\}
\end{equation*}
\begin{equation*}
T^{+}(n) = max\{|t|, h(t) = n\}
\end{equation*}
\end{definicao}

\begin{corolario}
\begin{equation*}
T^{-}\Big(h(t)\Big) \leq |t| \leq T^{+}\Big(h(t)\Big) \qquad
\forall t \in \mathbb{B}
\end{equation*}
\end{corolario}

\begin{teorema}
\begin{equation*}
T^{+}(n) = 2^n + 1
\end{equation*}
\end{teorema}

\begin{proof}
\begin{eqnarray*}
T^{+}(n) & = & max\{|t|, h(t) = n\} \\
& = & max\bigg\{\Big(|E(t)| + |D(t)| + 1 \Big), h(t) = n \bigg\} = \\
& = & 1 + max\{|t| + |t|, h(t) < n \} = \\
& = & 1 + max\{2|t|, h(t) < n \} = \\
& = & 1 + 2max\{|t|, h(t) < n \} = \\
& = & 1 + 2max\{T^{+}(k), k < n\} = \\
& = & 1 + 2T^{+}(n - 1) = \\
& = & 2^n - 1
\end{eqnarray*}
\end{proof}

\begin{corolario}
\begin{equation*}
T^{-}\Big(h(t)\Big) \leq |t| \leq 2^{h(t)} - 1 \qquad
\forall t \in \mathbb{B}
\end{equation*}
\end{corolario}

\begin{corolario}
Devido ao corol\'{a}rio anterior, temos:
\begin{equation*}
h(t) \geq \log(|t|+1)
\end{equation*}
Portanto:
\begin{equation*}
h(t) \geq \lceil \log(|t| + 1) \rceil = \lfloor \log(|t|) \rfloor + 1
\end{equation*}
\end{corolario}

\begin{corolario}
Devido aos corol\'{a}rios anteriores, temos:
\begin{equation*}
\lfloor \log(|t|) \rfloor + 1 \leq h(t) \leq |t|, \qquad
\forall t \in \mathbb{B}
\end{equation*}
\end{corolario}

\begin{definicao}
Definimos $\mathbb{AVL}$ como o conjunto de todas \'{a}rvores
$t \in \mathbb{B}$ que s\~{a}o vazias ou satisfazem: \\
\begin{itemize}
\item $E(t) \in \mathbb{AVL}$
\item $D(t) \in \mathbb{AVL}$
\item $\bigg|h\Big(E(t)\Big) - h\Big(D(t)\Big)\bigg| \leq 1$
\end{itemize}
Doravante, denominaremos de \'{a}rvore AVL qualquer \'{a}rvore
pertencente ao conjunto $\mathbb{AVL}$.
\end{definicao}

\begin{teorema}
\begin{equation*}
\lfloor \log|t| \rfloor + 1 \leq h(t) \leq 1.4405 \cdot \log|t|
\qquad \forall t \in \mathbb{AVL}
\end{equation*}
\end{teorema}

\begin{proof}
Seja $T_{AVL}^{-}(n)$ o tamanho m\'{i}nimo de uma \'{a}rvore AVL de
altura n.
\begin{eqnarray*}
T_{AVL}^{-}(n)&=& min\{|t|,h(t)=n \text{ e } t \in \mathbb{AVL}\}=\\
& = & 1+min \{|E(t)|+|D(t)|, h(t) = n \text{ e } t \in \mathbb{AVL}\}
\end{eqnarray*}
Temos que $h\big(E(t)\big) < h(t) = n$ e $h\big(D(t)\big) < h(t) = n$. \\
Supondo que $h\big(E(t)\big) = n - 1$, como $t$ \'{e} AVL, ent\~{a}o
$h\big(D(t)\big) = n - 1$ ou $h\big(D(t)\big) = n - 2$. Ent\~{a}o:
\begin{eqnarray*}
T_{AVL}^{-}(n)&=&1+min\{E(t),h(E(t))=n-1\text{ e } t \in \mathbb{AVL}\}+\\
&& +\: min\{D(t),h(D(t))=n-1\text{ e } t \in \mathbb{AVL}\} = \\
& = & 1 + min \{|t|, h(t) = n - 1 \text{ e } t \in \mathbb{AVL}\} +\\
&& +\: min \{|t|, h(t) = n - 2 \text{ e } t \in \mathbb{AVL}\} = \\
& = & 1 + t^{-}(n-1) + t^{-}(n-2)
\end{eqnarray*}
Cuja solu\c{c}\~{a}o quando $T_{AVL}^{-}(0) = 0$ \'{e}:
\begin{eqnarray*}
T_{AVL}^{-}(n)&=&\frac{5-3\sqrt{5}}{2}\Big(\frac{1-\sqrt{5}}{2}\Big)^n+
\frac{5+3\sqrt{5}}{2}\Big(\frac{1+\sqrt{5}}{2}\Big)^n - 1 = \\
& = & c_1r_1^n + c_2r_2^n - 1 = \\
&=&c_2r_2^n\Big(1+\frac{c_1r_1^n}{c_2r_2^n}-\frac{1}{c_2r_2^n}\Big)=\\
& = & f(n)r_2^n
\end{eqnarray*}
Como $t^{-}\big(h(t)\big)\leq|t|$ ent\~{a}o $\forall t \in \mathbb{AVL}$
$f\big(h(t)\big)r^{h(t)} \leq |t|$, ou seja:
\begin{equation*}
h(t)\log r_2 + \log f\big(h(t)\big) \leq \log |t|
\end{equation*}
Portanto:
\begin{equation*}
h(t) \leq \frac{\log |t| - \log f\big(h(t)\big)}{\log r_2} = 
\frac{1}{\log r_2} \log |t| - \frac{\log f\big(h(t)\big)}{\log r_2}
\end{equation*}
Como $f$ \'{e} crescente, $f(n) > 1$ para todo $n \geq 2$, portanto
$\log f(n) > 0$, ent\~{a}o:
\begin{equation*}
h(t) \leq \frac{1}{\log r_2} \log |t| =
\frac{1}{\log{\frac{1+\sqrt{5}}{2}}} \log |t| < 1.4405 \cdot \log |t|
\end{equation*}
Logo, se $h(t) \geq 2$ ent\~{a}o $h(t) \leq 1.4405 \cdot \log |t|$.
Ora, como $h(t) \geq \lfloor \log |t| \rfloor + 1$, ent\~{a}o:
\begin{equation*}
\lfloor \log|t| \rfloor + 1 \leq h(t) \leq 1.4405 \cdot \log|t|
\qquad \forall t \in \mathbb{AVL}
\end{equation*}
\end{proof}

\section{An\'{a}lise Assint\'{o}tica}

\begin{definicao}
Dadas $f,g: \mathbb{N} \longrightarrow \mathbb{R}$, dizemos que
$g$ \'{e} assint\'{o}ticamente limitada superiormente por $f$ se
existem $c_p \in \mathbb{C}$ e $n_p \in \mathbb{N}$ tais que
$c_p > 0$ e:
\begin{equation*}
|g(n)| \leq c_p|f(n)| \qquad \forall n \geq n_p
\end{equation*}
Escrevemos $g(n) \in O\big(f(n)\big)$. \\
Uma conven\c{c}\~{a}o de nota\c{c}\~{a}o estabelece que a
express\~{a}o $O\big(f(n)\big)$ do lado direito de uma
igualdade significa que a igualdade \'{e} satisfeita
substituindo-se $O\big(f(n)\big)$ por algum elemento
do conjutno $O\big(f(n)\big)$. \\
Tamb\'{e}m fica estabelecido que a express\~{a}o $O\big(f(n)\big)$
do lado esquerdo de uma igualdade indica que a igualdade \'{e}
satisfeita para todas as fun\c{c}\~{o}es do conjunto $O\big(f(n)\big)$.
\end{definicao}

\begin{exemplo}
Existe $f(n) \in O(1)$ tal que:
$\sum _{i=1}^{n}\frac{1}{i} = \ln(n) + f(n)$, portanto:
$\sum _{i=1}^{n}\frac{1}{i} - \ln(n)$ \'{e} assint\'{o}ticamente
limitada superiormente por uma constante.
\end{exemplo}

\begin{teorema}
\begin{equation*}
\log _b n  = O\big(\log n \big) \qquad \forall b > 1
\end{equation*}
\end{teorema}

\begin{exemplo}
Seja $t(n) = c_1+c_2n$, onde $c_1>0$ e $c_2>0$.
Como $|c_1| \leq c_1|1|$ ent\~{a}o $c_1 = O(1)$. An\'{a}logo para $c_2$.\\
Portanto: $t(n) = O(1) +O(1)n$. Temos que: \\
$c_1 + c_2n = \big(\frac{c_1}{n} + c_2\big)n$. Como:
\begin{equation*}
\Big| \frac{c_1}{n} + c_2\Big| \leq 2c_2 \qquad
\forall n > \frac{c_1}{c_2}
\end{equation*}
Ent\~{a}o $\frac{c_1}{n} + c_2 = O(1)$. Portanto $t(n)=O(1)n$. Temos que:
\begin{equation*}
f(n)=O(1) \Rightarrow  \exists c_p > 0, n_p \in \mathbb{N} \qquad
|f(n)| \leq c_p|1| \qquad \forall n \geq n_p
\end{equation*}
Multiplicando por $|n|$ temos:
\begin{equation*}
|n| |f(n)| \leq c_p|n|
\end{equation*}
Portanto $t(n) = O(n)$.
\end{exemplo}

\begin{teorema}
\begin{equation*}
\text{Se } \overline{f}(n) = O\big(f(n)\big) \text{ e } \overline{g}(n) =
O\big(g(n)\big) \text{ ent\~{a}o } \overline{f}(n)\overline{g}(n) =
O\big(f(n)g(n)\big)
\end{equation*}
\end{teorema}

\begin{teorema}
Reflexividade:
\begin{equation*}
f(n) = O \Big(f(n) \Big)
\end{equation*}
\end{teorema}

\begin{teorema}
\begin{equation*}
\text{Se } \overline{f}(n) = O \Big(f(n) \Big) \text{ e }
\overline{g}(n) = O \Big(g(n) \Big) \text{ ent\~{a}o }
\overline{f}(n) + \overline{g}(n) =
O\bigg(\Big|f(n)\Big| + \Big|g(n)\Big|\bigg)
\end{equation*}
\end{teorema}

\begin{corolario}
\begin{equation*}
O\Big(1\Big)O\Big(f(n)\Big) = O\Big(f(n)Big)
\end{equation*}
\end{corolario}

\begin{teorema}
Transitividade:
\begin{equation*}
\text{Se } h(n) = O\Big(f(n)\Big) \text{ e } f(n) = O\Big(g(n)\Big)
\text{ ent\~{a}o } h(n) = O\Big(g(n)\Big)
\end{equation*}
\end{teorema}

\end{document}
